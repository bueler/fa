\documentclass[12pt]{amsart}
%prepared in AMSLaTeX, under LaTeX2e
\addtolength{\oddsidemargin}{-.55in} 
\addtolength{\evensidemargin}{-.55in}
\addtolength{\topmargin}{-.4in}
\addtolength{\textwidth}{1.1in}
\addtolength{\textheight}{.6in}

\renewcommand{\baselinestretch}{1.05}

\usepackage{verbatim} % for "comment" environment

\usepackage{palatino}

\newtheorem*{thm}{Theorem}
\newtheorem*{defn}{Definition}
\newtheorem*{example}{Example}
\newtheorem*{problem}{Problem}
\newtheorem*{remark}{Remark}

\usepackage{fancyvrb,xspace,dsfont}

\usepackage[final]{graphicx}

% macros
\usepackage{amssymb}

\usepackage{hyperref}
\hypersetup{pdfauthor={Ed Bueler},
            pdfcreator={pdflatex},
            colorlinks=true,
            citecolor=blue,
            linkcolor=red,
            urlcolor=blue,
            }

\newcommand{\br}{\mathbf{r}}
\newcommand{\bv}{\mathbf{v}}
\newcommand{\bx}{\mathbf{x}}
\newcommand{\by}{\mathbf{y}}

\newcommand{\cF}{\mathcal{F}}
\newcommand{\cH}{\mathcal{H}}
\newcommand{\cL}{\mathcal{L}}
\newcommand{\cV}{\mathcal{V}}
\newcommand{\cW}{\mathcal{W}}

\newcommand{\CC}{\mathbb{C}}
\newcommand{\NN}{\mathbb{N}}
\newcommand{\RR}{\mathbb{R}}
\newcommand{\ZZ}{\mathbb{Z}}

\newcommand{\eps}{\epsilon}
\newcommand{\grad}{\nabla}
\newcommand{\lam}{\lambda}
\newcommand{\lap}{\triangle}

\newcommand{\ip}[2]{\ensuremath{\left<#1,#2\right>}}

\newcommand{\image}{\operatorname{im}}
\newcommand{\onull}{\operatorname{null}}
\newcommand{\rank}{\operatorname{rank}}
\newcommand{\range}{\operatorname{range}}
\newcommand{\trace}{\operatorname{tr}}

\newcommand{\prob}[1]{\bigskip\noindent\textbf{#1.}\quad }
\newcommand{\exer}[2]{\prob{Exercise #2 in Lecture #1}}

\newcommand{\pts}[1]{(\emph{#1 pts}) }
\newcommand{\epart}[1]{\medskip\noindent\textbf{(#1)}\quad }
\newcommand{\ppart}[1]{\,\textbf{(#1)}\quad }

\newcommand{\Matlab}{\textsc{Matlab}\xspace}
\newcommand{\Octave}{\textsc{Octave}\xspace}
\newcommand{\Python}{\textsc{Python}\xspace}
\newcommand{\Julia}{\textsc{Julia}\xspace}

\newcommand{\fl}{\operatorname{fl}}

\newcommand{\ds}{\displaystyle}

\DefineVerbatimEnvironment{mVerb}{Verbatim}{numbersep=2mm,
frame=lines,framerule=0.1mm,framesep=2mm,xleftmargin=4mm,fontsize=\footnotesize}

\newcommand{\nex}{\medskip\noindent}


\begin{document}
\scriptsize \noindent Math 617 Functional Analysis (Bueler) \hfill \emph{assigned 26 February 2024}
\normalsize\medskip

\Large\centerline{\textbf{Assignment 4}}
\large
\medskip

\centerline{\textbf{Due Wednesday 6 March 2024}}
\medskip
\normalsize

\thispagestyle{empty}

\bigskip
\noindent This Assignment is based primarily on sections 2.6, 2.7, and 3.1 of our textbook, D.~Borthwick (2020).~\emph{Spectral Theory: Basic Concepts and Applications}, Springer.

\medskip
\noindent \textsc{Please do the following exercises.}
\smallskip

\prob{P19}  \emph{This is Exercise 2.6, the proof of Corollary 2.29.  Hint: First show that for fixed $w\in\cH$, the functional $\ell(v) = \eta(w,v)$ is in $\cH'$.  Remember that $\ip{\cdot}{\cdot}$ and $\eta(\cdot,\cdot)$ are only conjugate linear in the first spot.  Please prove the linearity and boundedness of $T$.}

\medskip\noindent Given a bounded sesquilinear form $\eta:\cH\times \cH \to \CC$, show that there is a unique operator $T\in \cL(\cH)$ so that
	$$\eta(v,w) = \ip{v}{Tw}$$
for all $v,w\in\cH$.


\prob{P20}  \emph{This is Exercise 2.10, but clarified.  One way to show \emph{\textbf{(a)}} is to argue that it is $\range(P)=\ker(Q)$ of some operator $Q$ built from $P$.}

\medskip\noindent Suppose $P\in\cL(\cH)$ satisfies $P^2=P$ and $\ip{Pv}{w} = \ip{v}{Pw}$ for all $v,w\in\cH$.

\epart{a}  Show that $\range(P)$ is a closed subspace of $\cH$.

\epart{b}  Show that
	$$\cH = \range(P) \oplus \ker(P).$$


\prob{P21}  Carefully prove Corollary 2.36, which is Parseval's theorem and equality, from Theorems 2.34 and 2.35.


\prob{P22}  \emph{Hints.  Cauchy-Schwarz for \emph{\textbf{(a)}}, naturally.  An example for \emph{\textbf{(b)}} can be built as a power of the polar coordinate $r=(x^2+y^2)^{1/2}$.}

\epart{a}  Let $X=(0,1)^2 \subset \RR^2$.  Suppose $K \in L^2(X)$, that is,
	$$\int_0^1 \int_0^1 |K(x,y)|^2\,dx\,dy < \infty.$$
Show that
	$$(Tf)(x) = \int_0^1 K(x,y) f(y)\,dy$$
defines $T\in\cL(\cH)$ for $\cH = L^2(0,1)$.

\epart{b} Find $K \in L^2(X)$ which is \emph{not} a bounded function on $X$.


\prob{P23}  FIXME Dirichlet kernel stuff


\end{document}
