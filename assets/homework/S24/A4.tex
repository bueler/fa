\documentclass[12pt]{amsart}
%prepared in AMSLaTeX, under LaTeX2e
\addtolength{\oddsidemargin}{-.55in} 
\addtolength{\evensidemargin}{-.55in}
\addtolength{\topmargin}{-.4in}
\addtolength{\textwidth}{1.1in}
\addtolength{\textheight}{.6in}

\renewcommand{\baselinestretch}{1.05}

\usepackage{verbatim} % for "comment" environment

\usepackage{palatino}

\newtheorem*{thm}{Theorem}
\newtheorem*{defn}{Definition}
\newtheorem*{example}{Example}
\newtheorem*{problem}{Problem}
\newtheorem*{remark}{Remark}

\usepackage{fancyvrb,xspace,dsfont}

\usepackage[final]{graphicx}

% macros
\usepackage{amssymb}

\usepackage{hyperref}
\hypersetup{pdfauthor={Ed Bueler},
            pdfcreator={pdflatex},
            colorlinks=true,
            citecolor=blue,
            linkcolor=red,
            urlcolor=blue,
            }

\newcommand{\br}{\mathbf{r}}
\newcommand{\bv}{\mathbf{v}}
\newcommand{\bx}{\mathbf{x}}
\newcommand{\by}{\mathbf{y}}

\newcommand{\cF}{\mathcal{F}}
\newcommand{\cH}{\mathcal{H}}
\newcommand{\cL}{\mathcal{L}}
\newcommand{\cV}{\mathcal{V}}
\newcommand{\cW}{\mathcal{W}}

\newcommand{\CC}{\mathbb{C}}
\newcommand{\NN}{\mathbb{N}}
\newcommand{\RR}{\mathbb{R}}
\newcommand{\ZZ}{\mathbb{Z}}

\newcommand{\eps}{\epsilon}
\newcommand{\grad}{\nabla}
\newcommand{\lam}{\lambda}
\newcommand{\lap}{\triangle}

\newcommand{\ip}[2]{\ensuremath{\left<#1,#2\right>}}

\newcommand{\image}{\operatorname{im}}
\newcommand{\onull}{\operatorname{null}}
\newcommand{\rank}{\operatorname{rank}}
\newcommand{\range}{\operatorname{range}}
\newcommand{\trace}{\operatorname{tr}}

\newcommand{\Span}{\operatorname{span}}

\newcommand{\prob}[1]{\bigskip\noindent\textbf{#1.}\quad }
\newcommand{\exer}[2]{\prob{Exercise #2 in Lecture #1}}

\newcommand{\pts}[1]{(\emph{#1 pts}) }
\newcommand{\epart}[1]{\medskip\noindent\textbf{(#1)}\quad }
\newcommand{\ppart}[1]{\,\textbf{(#1)}\quad }

\newcommand{\Matlab}{\textsc{Matlab}\xspace}
\newcommand{\Octave}{\textsc{Octave}\xspace}
\newcommand{\Python}{\textsc{Python}\xspace}
\newcommand{\Julia}{\textsc{Julia}\xspace}

\newcommand{\fl}{\operatorname{fl}}

\newcommand{\ds}{\displaystyle}

\DefineVerbatimEnvironment{mVerb}{Verbatim}{numbersep=2mm,
frame=lines,framerule=0.1mm,framesep=2mm,xleftmargin=4mm,fontsize=\footnotesize}

\newcommand{\nex}{\medskip\noindent}


\begin{document}
\scriptsize \noindent Math 617 Functional Analysis (Bueler) \hfill \emph{assigned 26 February 2024}
\normalsize\medskip

\Large\centerline{\textbf{Assignment 4}}
\large
\medskip

\centerline{\textbf{Due Wednesday 6 March 2024}}
\medskip
\normalsize

\thispagestyle{empty}

\bigskip
\noindent This Assignment is based primarily on sections 2.6, 2.7, and 3.1 of our textbook, Borthwick (2020)~\emph{Spectral Theory: Basic Concepts and Applications}, Springer.

\medskip
\noindent \textsc{Please do the following exercises.}
\smallskip


\prob{P19}  \emph{This is Exercise 2.10, but clarified.  One way to show \emph{\textbf{(a)}} is to argue that $\range(P)=\ker(Q)$ for some bounded operator $Q\in\cL(\cH)$ built from $P$.}

\medskip\noindent Suppose $P\in\cL(\cH)$ satisfies $P^2=P$ and $\ip{Pv}{w} = \ip{v}{Pw}$ for all $v,w\in\cH$.

\epart{a}  Show that $\range(P)$ is a closed subspace of $\cH$.

\epart{b}  Show that $\cH = \range(P) \oplus \ker(P)$.

\prob{P20}  \emph{This is Exercise 2.6, asking for the proof of Corollary 2.29.  Hint: First show that for fixed $w\in\cH$, the functional $\ell(v) = \eta(w,v)$ is in $\cH'$.  Remember that $\ip{\cdot}{\cdot}$ and $\eta(\cdot,\cdot)$ are only conjugate linear in the first spot.  Please prove the linearity and boundedness of $T$.}

\medskip\noindent Given a bounded sesquilinear form $\eta:\cH\times \cH \to \CC$, show that there is a unique operator $T\in \cL(\cH)$ so that
	$$\eta(v,w) = \ip{v}{Tw}$$
for all $v,w\in\cH$.


\prob{P21}  Carefully prove Corollary 2.36, which is Parseval's theorem and equality, from Theorems 2.34 and 2.35.


\prob{P22}  \emph{In Example 2.32 on page 28 our textbook casually says that a ``standard argument using the Dirichlet kernel'' shows that a certain set of orthogonal complex exponentials is in fact a basis.  This exercise begins this quite substantial ``standard argument,'' actually perhaps the greatest triumph of 19th century analysis, by defining the Dirichlet kernel.  Note I use $\cH=L^2(-\pi,\pi)$ here, instead of $L^2(0,2\pi)$, but this is a detail.}

\epart{a}  Let $\ds \phi_k(\theta) = \frac{1}{\sqrt{2 \pi}} e^{ik\theta}$ for $k\in\ZZ$.  Since these functions are continuous and bounded they are in $\cH=L^2(-\pi,\pi)$.  Show that $\{\phi_k\}_{k\in\ZZ}$ is an orthonormal set.

\epart{b}  For $n\in\NN$, let $\ds D_n(\theta) = \sum_{j=-n}^n e^{ij\theta}$ be the Dirichlet kernel.  Show that $D_n(0)=2n+1$ (\emph{easy!}).  For $\theta\ne 0$ show by a geometric series argument that
	$$D_n(\theta) = \frac{\sin\left((n+1/2)\theta\right)}{\sin(\theta/2)}.$$

\epart{c}  Use a computer to plot $D_2(\theta)$, $D_8(\theta)$, and $D_{20}(\theta)$ in a single figure, on the interval $-\pi \le \theta \le \pi$.

\epart{d}  Compute $\ds \int_{-\pi}^\pi D_n(\theta)\,d\theta$.  Conjecture what the integral $\ds \int_{-\pi}^\pi D_n(\theta) f(\theta)\,d\theta$ yields when $f(x)$ is continuous at $x=0$, and if $n$ is very large?

\epart{e}  Show that if $f\in\cH=L^2(-\pi,\pi)$ then
	$$\int_{-\pi}^\pi D_n(\theta - x) f(\theta)\,d\theta = 2\pi \sum_{j=-n}^n \ip{\phi_j}{f} \phi_j(x).$$
Observe---there is nothing to prove here---that this says that
	$$(P_n f)(x) = \frac{1}{2\pi} \int_{-\pi}^\pi D_n(\theta - x) f(\theta)\,d\theta$$
is the orthogonal projection of $f \in \cH$ onto $\Span\{\phi_{-n},\dots,\phi_n\}$; compare (2.30).


\prob{P23}  \emph{Hints.  Use Cauchy-Schwarz for part \emph{\textbf{(a)}}.  An example for part \emph{\textbf{(b)}} can be built from a power of the polar coordinate $r=(x^2+y^2)^{1/2}$.}

\epart{a}  Let $\Omega=(0,1)^2 \subset \RR^2$.  Suppose $K \in L^2(\Omega)$, that is,
	$$\int_0^1 \int_0^1 |K(x,y)|^2\,dx\,dy < \infty.$$
Show that the integral operator with kernel $K$, namely,
	$$(T_K f)(x) = \int_0^1 K(x,y) f(y)\,dy$$
is linear and bounded ($T_K \in\cL(\cH)$) on $\cH = L^2(0,1)$.

\epart{b} Find $K \in L^2(\Omega)$ which is \emph{not} a bounded function.

\epart{c} For a finite sequence $a_j\in\CC$, $j=1,\dots,n$, let
    $$K_{a}(x,y) = \sum_{j=1}^n a_j e^{i \,2\pi j(x-y)}.$$
This is a continuous and bounded function and thus $K_a \in L^2(\Omega)$.  Find all of the eigenvalues and eigenfunctions of the operator $T_{K_a} \in \cL(\cH)$.

\end{document}
