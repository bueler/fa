\documentclass[12pt]{amsart}
%prepared in AMSLaTeX, under LaTeX2e
\addtolength{\oddsidemargin}{-.55in} 
\addtolength{\evensidemargin}{-.55in}
\addtolength{\topmargin}{-.4in}
\addtolength{\textwidth}{1.1in}
\addtolength{\textheight}{.6in}

\renewcommand{\baselinestretch}{1.05}

\usepackage{verbatim} % for "comment" environment

\usepackage{palatino}

\newtheorem*{thm}{Theorem}
\newtheorem*{defn}{Definition}
\newtheorem*{example}{Example}
\newtheorem*{problem}{Problem}
\newtheorem*{remark}{Remark}

\usepackage{fancyvrb,xspace,dsfont}

\usepackage[final]{graphicx}

% macros
\usepackage{amssymb}

\usepackage{hyperref}
\hypersetup{pdfauthor={Ed Bueler},
            pdfcreator={pdflatex},
            colorlinks=true,
            citecolor=blue,
            linkcolor=red,
            urlcolor=blue,
            }

\newcommand{\br}{\mathbf{r}}
\newcommand{\bv}{\mathbf{v}}
\newcommand{\bx}{\mathbf{x}}
\newcommand{\by}{\mathbf{y}}

\newcommand{\cF}{\mathcal{F}}
\newcommand{\cH}{\mathcal{H}}
\newcommand{\cL}{\mathcal{L}}
\newcommand{\cV}{\mathcal{V}}
\newcommand{\cW}{\mathcal{W}}

\newcommand{\CC}{\mathbb{C}}
\newcommand{\NN}{\mathbb{N}}
\newcommand{\RR}{\mathbb{R}}
\newcommand{\ZZ}{\mathbb{Z}}

\newcommand{\eps}{\epsilon}
\newcommand{\grad}{\nabla}
\newcommand{\lam}{\lambda}
\newcommand{\lap}{\triangle}

\newcommand{\ip}[2]{\ensuremath{\left<#1,#2\right>}}

\newcommand{\image}{\operatorname{im}}
\newcommand{\onull}{\operatorname{null}}
\newcommand{\rank}{\operatorname{rank}}
\newcommand{\range}{\operatorname{range}}
\newcommand{\trace}{\operatorname{tr}}

\newcommand{\prob}[1]{\bigskip\noindent\textbf{#1.}\quad }
\newcommand{\exer}[2]{\prob{Exercise #2 in Lecture #1}}

\newcommand{\pts}[1]{(\emph{#1 pts}) }
\newcommand{\epart}[1]{\medskip\noindent\textbf{(#1)}\quad }
\newcommand{\ppart}[1]{\,\textbf{(#1)}\quad }

\newcommand{\Matlab}{\textsc{Matlab}\xspace}
\newcommand{\Octave}{\textsc{Octave}\xspace}
\newcommand{\Python}{\textsc{Python}\xspace}
\newcommand{\Julia}{\textsc{Julia}\xspace}

\newcommand{\fl}{\operatorname{fl}}

\newcommand{\ds}{\displaystyle}

\DefineVerbatimEnvironment{mVerb}{Verbatim}{numbersep=2mm,
frame=lines,framerule=0.1mm,framesep=2mm,xleftmargin=4mm,fontsize=\footnotesize}

\newcommand{\nex}{\medskip\noindent}


\begin{document}
\scriptsize \noindent Math 617 Functional Analysis (Bueler) \hfill \emph{assigned 7 February 2024}
\normalsize\medskip

\Large\centerline{\textbf{Assignment 3}}
\large
\medskip

\centerline{\textbf{Due Wednesday 21 February 2024}}
\medskip
\normalsize

\thispagestyle{empty}

\bigskip
\noindent This Assignment is based primarily sections 2.3, 2.4, 2.6, 2.7, and 3.1 of our textbook, D.~Borthwick (2020).~\emph{Spectral Theory: Basic Concepts and Applications}, Springer.

\medskip
\noindent \textsc{Do the following exercises.}
\smallskip

\prob{P13}  \emph{Note that the polarization identity (2.17) in the textbook has a typo.  I believe that the one below is correct.  In any case, be careful with which inner product argument has conjugate linearity and which has ordinary linearity!}

\epart{a} Let $(V,\ip{\cdot}{\cdot})$ be an inner product space.  Recall that $\|v\|=\sqrt{\ip{v}{v}}$ defines the norm.  Prove the parallelogram law
    $$\|u+v\|^2 + \|u-v\|^2 = 2 \|u\|^2 + 2 \|v\|^2.$$

\epart{b} For a complex inner product space, prove the polarization identity
    $$\ip{u}{v} = \frac{1}{4} \Big(\|u+v\|^2 - \|u-v\|^2 - i\|u+iv\|^2 + i\|u-iv\|^2\Big).$$


\prob{P14}  \emph{I want you to be able to successfully \emph{read} Chapter 3 as soon as possible.  This and later Chapters are typical of post-1960 mathematical writing in analysis, mathematical physics, or partial differential equations.  Specifically, they assume familiarity with many vector spaces of functions. The precise count in the ``name game'' of this problem is not quite the issue, but please identify all the major vector spaces at least.  For instance, in my count I do not include ``$\cH_1$'' and ``$\cH_2$'', when they appear in the middle of page 36,  because they are generic Hilbert spaces, and ``$\cH$'' has already been mentioned.}

\medskip
\noindent
Section 3.1 is only two pages long.  However, by my count, 15 different vector spaces are mentioned.  Most are complex vector space (but not all) and most are spaces of complex-valued functions (but not all).  Some are subspaces of others.  In a numbered list, identify all the distinct vector spaces mentioned in section 3.1, and for each one write out the definition.  State all subspace relationships.


\prob{P15}  \ppart{a} For each $k\in \NN \cup\{\infty\}$, give an example of a linear isometry $A$ on $\ell^2$ for which $(\range{A})^\perp$ has dimension $k$.  (\emph{Hint: Shifts.})

\epart{b}  For each $k\in \NN \cup\{\infty\}$, give an example of a linear map on $\ell^2$ for which the kernel has dimension $k$.  (\emph{Hint: Shifts.})


\clearpage\newpage
\prob{P16}  \emph{Multiplication operators are defined in Example 2.8 on page 10.}

\medskip
\noindent
Consider $L^2(0,1)$ and fix $f=\chi_{(0,1/2)}=\mathds{1}_{(0,1/2)}$, which is the characteristic function of $I=(0,1/2)$.  What is $\|M_f\|$?  Show that $M_f$ has eigenvalues 0 and 1.  What are the geometric multiplicities of these eigenvalues?


\prob{P17}  \emph{Consider the smooth functions and the compact-support subspace:}
\begin{align*}
C^\infty(\RR) &= \left\{f:\RR\to\CC\,\big|\, f^{(n)} \in C(\RR) \text{ for } n=0,1,2,\dots\right\} \\
C_0^\infty(\RR) &= \left\{f\in C^\infty(\RR)\big|\, \text{there exists $I=[a,b]$ so that $f(x)=0$ for $x\notin I$}\right\}
\end{align*}
\emph{In this problem we see that functions in $C^\infty(\RR)$ are nice, but they are not as nice as $C_0^\infty(\RR)$.}

\epart{a} Show that $C_0^\infty(\RR) \subseteq L^2(\RR)$, but that $C^\infty(\RR) \nsubseteq L^2(\RR)$.

\epart{b} Find a function $f \in \left(C^\infty(\RR) \cap L^2(\RR)\right) \setminus C_0^\infty(\RR)$ for which every derivative $f^{(n)}$ is also in $L^2(\RR)$.  (\emph{Recommended search technique: Consider reasonably familiar functions.})

\prob{P18}  \emph{The main importance of the Fourier transform is that it converts differentiation into multiplication by polynomials.  One needs integration by parts to prove this, but it encounters no difficulties here because you only prove the formulas for $f\in C_0^\infty(\RR)$.}

\medskip\noindent
From the Fourier transform of $f$ defined by formula (2.18) on page 17, denoted $\cF(f)(\xi)=\hat f(\xi)$, prove that for $f\in C_0^\infty(\RR)$ we have
\begin{align*}
\cF(f')(\xi) &= i\xi \hat f(\xi) \\
\cF(f'')(\xi) &= -\xi^2 \hat f(\xi)
\end{align*}


\end{document}
