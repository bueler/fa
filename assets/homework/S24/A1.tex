\documentclass[12pt]{amsart}
%prepared in AMSLaTeX, under LaTeX2e
\addtolength{\oddsidemargin}{-.55in} 
\addtolength{\evensidemargin}{-.55in}
\addtolength{\topmargin}{-.4in}
\addtolength{\textwidth}{1.1in}
\addtolength{\textheight}{.6in}

\renewcommand{\baselinestretch}{1.05}

\usepackage{verbatim} % for "comment" environment

\usepackage{palatino}

\newtheorem*{thm}{Theorem}
\newtheorem*{defn}{Definition}
\newtheorem*{example}{Example}
\newtheorem*{problem}{Problem}
\newtheorem*{remark}{Remark}

\usepackage{fancyvrb,xspace}

\usepackage[final]{graphicx}

% macros
\usepackage{amssymb}

\usepackage{hyperref}
\hypersetup{pdfauthor={Ed Bueler},
            pdfcreator={pdflatex},
            colorlinks=true,
            citecolor=blue,
            linkcolor=red,
            urlcolor=blue,
            }

\newcommand{\br}{\mathbf{r}}
\newcommand{\bv}{\mathbf{v}}
\newcommand{\bx}{\mathbf{x}}
\newcommand{\by}{\mathbf{y}}

\newcommand{\CC}{\mathbb{C}}
\newcommand{\NN}{\mathbb{N}}
\newcommand{\RR}{\mathbb{R}}
\newcommand{\ZZ}{\mathbb{Z}}

\newcommand{\eps}{\epsilon}
\newcommand{\grad}{\nabla}
\newcommand{\lam}{\lambda}
\newcommand{\lap}{\triangle}

\newcommand{\ip}[2]{\ensuremath{\left<#1,#2\right>}}

\newcommand{\image}{\operatorname{im}}
\newcommand{\onull}{\operatorname{null}}
\newcommand{\rank}{\operatorname{rank}}
\newcommand{\range}{\operatorname{range}}
\newcommand{\trace}{\operatorname{tr}}

\newcommand{\prob}[1]{\bigskip\noindent\textbf{#1.}\quad }
\newcommand{\exer}[2]{\prob{Exercise #2 in Lecture #1}}

\newcommand{\pts}[1]{(\emph{#1 pts}) }
\newcommand{\epart}[1]{\medskip\noindent\textbf{(#1)}\quad }
\newcommand{\ppart}[1]{\,\textbf{(#1)}\quad }

\newcommand{\Matlab}{\textsc{Matlab}\xspace}
\newcommand{\Octave}{\textsc{Octave}\xspace}
\newcommand{\Python}{\textsc{Python}\xspace}
\newcommand{\Julia}{\textsc{Julia}\xspace}

\newcommand{\fl}{\operatorname{fl}}

\newcommand{\ds}{\displaystyle}

\DefineVerbatimEnvironment{mVerb}{Verbatim}{numbersep=2mm,
frame=lines,framerule=0.1mm,framesep=2mm,xleftmargin=4mm,fontsize=\footnotesize}


\begin{document}
\scriptsize \noindent Math 617 Functional Analysis (Bueler) \hfill \emph{assigned 17 January 2024}
\normalsize\medskip

\Large\centerline{\textbf{Assignment 1}}
\large
\medskip

\centerline{\textbf{Due Friday 26 January 2024, at the start of class} (\emph{revised})}
\medskip
\normalsize

\thispagestyle{empty}

\bigskip
\noindent This Assignment is based primarily on the ``Definitions and facts'' handout, but also on sections 2.1 and 2.2 of our textbook

\begin{quote}
D.~Borthwick (2020).~\emph{Spectral Theory: Basic Concepts and Applications}, Springer
\end{quote}

\medskip
\noindent \textsc{Do the following exercises.}
\smallskip

%\prob{P1}  \emph{FIXME}

\prob{P1}  Suppose $(V,\|\cdot\|)$ is a normed vector space.  Let $Y \subset V$ be any finite set of points.  Show that $Y$ is closed.

\prob{P2}  Let $(V,\|\cdot\|)$ be a normed vector space.  Let $S=\{x\in V\,:\,\|x\|=1\}$.  Recalling that $B_1$ is the open ball of radius one, show $S$ is exactly the boundary set of $B_1$: $\partial B_1=S$.

\prob{P3}  Let $X_k$ be the set of RNA sequences of length $k$.  All you need to know about such things is that an RNA sequence has one character from $\{A,C,G,U\}$ in each location.  That is, an element of $X_k$ is a string of $k$ letters, each of which is $A,C,G,U$.  Let $\delta(a,b)=0$ if $a=b$ and $\delta(a,b)=1$ if $a\ne b$.  For $x,y \in X_k$ let
	$$d(x,y) = \sum_{j=1}^k \delta(x_j,y_j)$$
where $x_j$ is the $j$th letter in the RNA sequence $x$.

\epart{a}  Show rigorously that $(X_k,d)$ is a metric space.  That is, show $d(x,y)$ computes a valid distance between any two RNA sequences of the same length.

\epart{b}  Can $(X_k,d)$ be regarded as a subset of a real normed vector space $(V,\|\cdot\|)$?  Argue, at least informally, that this is true.  You will need to construct $V$ and sketch the embedding.  Formally, one says that there is an \emph{isometric} embedding $X_k \hookrightarrow V$.  (\emph{Hint.}  I used $\|\cdot\|_\infty$ on each letter.)

\prob{P4}  Show that a convergent sequence in a metric space cannot converge to two different limits.

\prob{P5}  \ppart{a}  Is $\ds f(x) = \frac{1}{\sqrt{x}}$ in $L^1(0,1)$?  In $L^2(0,1)$?

\epart{b}  Let $s \ge 0$ and define $\ds g_s(x) = \frac{1}{x^s}$.  For $1\le p \le \infty$, determine exactly which spaces $L^p(0,1)$ contain $g_s$.

\prob{P6}  \emph{The definition of the spaces $\ell^p = \ell^p(\NN)$ in section 2.2 of the textbook is too terse.  This problem at least includes a clear definition for $p=\infty$.  Make sure to observe that infinite sequences are the same as functions from $\NN$ to $\CC$.}

\medskip
\noindent The normed vector space $(\ell^\infty,\|\cdot\|_\infty)$ is the set of infinite sequences $a=(a_k)=(a_1,a_2,a_3,\dots)$ with complex entries ($a_k\in\CC$) which are bounded, that is, so that there is $M>0$ so that $|a_k|<M$ for all $k \in\NN$.  The norm is the supremum of the absolute values of the entries:
	$$\|a\|_\infty = \sup_{k=1,2,\dots} |a_k|.$$

\epart{a}  Show that $(\ell^\infty,\|\cdot\|_\infty)$ is a Banach space.

\epart{b}  Consider the set of complex-valued sequences which have a limit of zero:
	$$Y=\{(a_k)\,:\, a_k \to 0\}.$$
Show that $Y \subset \ell^\infty$.

\epart{c}  Show that $(Y,\|\cdot\|_\infty)$ is a Banach space.  (\emph{Hint.}  You may do this directly or use part \textbf{(a)}.  For less than obvious reasons, this subset $Y$ is often instead called $c_0$.)

\prob{P7}  For any set $X$ and measure $\mu$, the normed vector space $L^1(X,d\mu)$ is defined in section 2.2 of our textbook.  For such a space the Minkowski (triangle) inequality (2.5) can be proved directly, without reference to Appendix A.2.  Do so.

\end{document}
