\documentclass[12pt]{amsart}
%prepared in AMSLaTeX, under LaTeX2e
\addtolength{\oddsidemargin}{-.55in} 
\addtolength{\evensidemargin}{-.55in}
\addtolength{\topmargin}{-.4in}
\addtolength{\textwidth}{1.1in}
\addtolength{\textheight}{.6in}

\renewcommand{\baselinestretch}{1.05}

\usepackage{verbatim} % for "comment" environment

\usepackage{palatino}

\newtheorem*{thm}{Theorem}
\newtheorem*{defn}{Definition}
\newtheorem*{example}{Example}
\newtheorem*{problem}{Problem}
\newtheorem*{remark}{Remark}

\usepackage{fancyvrb,xspace,dsfont}

\usepackage[final]{graphicx}

% macros
\usepackage{amssymb}

\usepackage{hyperref}
\hypersetup{pdfauthor={Ed Bueler},
            pdfcreator={pdflatex},
            colorlinks=true,
            citecolor=blue,
            linkcolor=red,
            urlcolor=blue,
            }

\newcommand{\br}{\mathbf{r}}
\newcommand{\bv}{\mathbf{v}}
\newcommand{\bx}{\mathbf{x}}
\newcommand{\by}{\mathbf{y}}

\newcommand{\cD}{\mathcal{D}}
\newcommand{\cF}{\mathcal{F}}
\newcommand{\cH}{\mathcal{H}}
\newcommand{\cL}{\mathcal{L}}
\newcommand{\cV}{\mathcal{V}}
\newcommand{\cW}{\mathcal{W}}

\newcommand{\CC}{\mathbb{C}}
\newcommand{\NN}{\mathbb{N}}
\newcommand{\RR}{\mathbb{R}}
\newcommand{\ZZ}{\mathbb{Z}}

\renewcommand{\Im}{\mathrm{Im}}
\renewcommand{\Re}{\mathrm{Re}}

\newcommand{\eps}{\epsilon}
\newcommand{\grad}{\nabla}
\newcommand{\lam}{\lambda}
\newcommand{\lap}{\triangle}

\newcommand{\ip}[2]{\ensuremath{\left<#1,#2\right>}}

\newcommand{\image}{\operatorname{im}}
\newcommand{\onull}{\operatorname{null}}
\newcommand{\rank}{\operatorname{rank}}
\newcommand{\range}{\operatorname{range}}
\newcommand{\trace}{\operatorname{tr}}
\newcommand{\Span}{\operatorname{span}}

\newcommand{\prob}[1]{\bigskip\noindent\textbf{#1.}\quad }

\newcommand{\pts}[1]{(\emph{#1 pts}) }
\newcommand{\epart}[1]{\medskip\noindent\textbf{(#1)}\quad }
\newcommand{\ppart}[1]{\,\textbf{(#1)}\quad }

\newcommand{\ds}{\displaystyle}

\newcommand{\nex}{\medskip\noindent}


\begin{document}
\scriptsize \noindent Math 617 Functional Analysis (Bueler) \hfill \emph{version 5; assigned 10 April 2024}
\normalsize\medskip

\Large\centerline{\textbf{Assignment 7}}
\large
\medskip

\centerline{\textbf{Due Monday 22 April 2024 (\emph{revised!})}}
\medskip
\normalsize

\thispagestyle{empty}

\bigskip
\noindent This Assignment is based on sections 3.2, 3.3, 3.4, 4.1, and 4.2 of our textbook, Borthwick (2020)~\emph{Spectral Theory: Basic Concepts and Applications}, Springer.

\medskip
\noindent \textsc{Please do the following exercises.}
\smallskip

\renewcommand{\SS}{\mathbb{S}}

\prob{P33}  The adjoint of a linear map between complex Hilbert spaces $\cH_1$ and $\cH_2$ can be defined.  Here we consider bounded operators only.  For $T \in \cL(\cH_1,\cH_2)$ the \emph{(Hilbert space) adjoint} $T^*$ is the unique linear map $T^* \in \cL(\cH_2,\cH_1)$ so that
	$$\ip{v}{Tu}_2 = \ip{T^*v}{u}_1 \qquad \text{for all $u\in\cH_1$ and $v\in\cH_2$}.$$
When $\cH_1=\cH_2$ this is the same definition as in section 3.2.

\epart{a}  Show that $(T^*)^* = T$ and that $(ST)^* = T^* S^*$.

\epart{b}  Show that if $T$ is invertible with $T^{-1}\in \cL(\cH_2,\cH_1)$ then $(T^*)^{-1} = (T^{-1})^*$.

\epart{c}  Suppose that $Q \in \cL(\cH_1,\cH_2)$ satisfies $Q^* Q = I_1$ and $Q Q^* = I_2$ where $I_i$ is the identity map on $\cH_i$.  Show that $Q$ is unitary.

\medskip \noindent {\small \emph{Hints.}  For part \textbf{(a)} assume that $S \in \cL(\cH_2,\cH_3)$.  For part \textbf{(b)} use $T T^{-1} = I_2$ and $T^{-1} T = I_1$, and then apply \textbf{(a)}.  For part \textbf{(c)} use the definition on page 17 of the text: $U$ is \emph{unitary} if it is a bijective isometry.}


\prob{P34}  \ppart{a}  Suppose $\{\phi_n\}$ is an orthonormal basis of a complex Hilbert space $\cH$.  Define the map $Q\in\cL(\cH,\ell^2)$, where $\ell^2=\ell^2(\NN)$, by $(Qf)_n = \ip{\phi_n}{f}_{\cH}$ for $f\in\cH$.  Give a formula for $Q^*$.  Show that $Q$ is unitary.

\epart{b}  Let $T$ be a closed (unbounded) linear operator on $\cH$.  Suppose $\phi_n \in \cD(T)$ and $T \phi_n = \lambda_n \phi_n$, for $n\in\NN$ and $\lambda_n\in\CC$.  If $\{\phi_n\}$ is an orthonormal basis of $\cH$ then $Q$ in part \textbf{(a)} unitarily diagonalizes $T$ in the sense that
	$$Q T Q^* = M$$
defines an unbounded multiplication operator on $\ell^2$.

\medskip \noindent {\small \emph{Hints.}  For part \textbf{(a)} you may use \textbf{P33(c)}, though that is not the only way.  For part \textbf{(b)}, make sure to define the domain of $M$ and the action of $M$ on elements of $\cD(M)$.}


\prob{P35}  \ppart{a}  Let $\cH = L^2(\RR)$.  Define $\left(M_{x^2}\, v\right)(x) = x^2 v(x)$, an unbounded multiplication operator with domain $\cD(M_{x^2}) = \{v\in\cH\,:\,x^2 v(x) \in \cH\}$.  Define $(Tv)(x) = v''(x)$, an unbounded second derivative operator with domain $\cD(T) = C_0^\infty(\RR)$.  Show that these operators have no eigenvalues.

\epart{b}  Let $\cH = L^2(0,\pi)$.  Define $\left(M_{x^2}\, v\right)(x) = x^2 v(x)$, a multiplication operator with domain $\cD(M_{x^2}) = \{v\in\cH\,:\,x^2 v(x) \in \cH\}$.  Show that $M_{x^2}$ is actually bounded, but that it has no eigenvalues.

\epart{c}  Let $\cH = L^2(0,\pi)$.  Define $(Tv)(x) = v''(x)$, a second derivative operator with domain $\cD(T) = \{v\in\cH\,:\,v \in C^2[0,\pi] \text{ and } v(0)=v(\pi)=0\}$.  Show that $T$ is unbounded, and that $\phi_k(x) = \sin(kx)$ is an eigenfunction for any $k\in\NN$. Find the corresponding eigenvalues.

\medskip \noindent {\small \emph{Hints.}  For part \textbf{(a)} you may use results in Example 3.3.  For part \textbf{(b)} you may use the result in Example 2.8.}

\medskip \noindent {\small \emph{Comments.}  You do not need to prove self-adjointness or spectrum.  However, textbook Examples 3.2, 3.5, and 3.22 show both $M_{x^2}$ are self-adjoint.  Example 3.26 shows that $T$ in part \textbf{(a)} is essentially self-adjoint.  Example 3.20 sketches why $T$ in part \textbf{(c)} is essentially self-adjoint.  See Theorems 4.5 for the spectrum of both $M_{x^2}$, thus by unitary-equivalence for the closure of $T$ in part \textbf{(a)} also.  Use \textbf{P34(b)} for the spectrum of the closure of $T$ in part \textbf{(b)}.}


\prob{P36}  Let $\cH$ be a complex Hilbert space.  Recall that if $A$ is a symmetric operator on $\cH$ then $v\in\cD(A)$ implies $\ip{v}{Av} \in \RR$.  We will write $A-z$ for $A-zI$.

\epart{a}  Suppose $A$ is a symmetric operator on $\cH$.  Show that if $z\in\CC$ then
	$$\Im \ip{v}{(A-z)v} = - \Im(z) \|v\|^2.$$

\epart{b}  If furthermore $z\in\CC$ is strictly complex, i.e.~$\Im z \ne 0$, then
	$$\|v\| \le \frac{\|(A-z)v\|}{|\Im(z)|}.$$
In this situation, show that $A-z$ is injective.


\prob{P37}  Let $\cH$ be a complex Hilbert space.  Recall that $\cL(\cH)$ is a normed vector space with norm $\|T\| = \sup_{\|v\|=1} \|Tv\|$, and also recall Theorem 2.10.

\epart{a}  Suppose that $T\in\cL(\cH)$, $z\in\CC$, and $|z| > \|T\|$.  Show that
	$$\sum_{k=0}^\infty z^{-k} T^k$$
converges to $S \in \cL(\cH)$.

\epart{b}  Under the same assumptions, show that
	$$S (T-z I) = (T-z I) S = - z I.$$
Explain why this shows $z \in \rho(T)$.


%\prob{P3X}  show $\sigma(U) \subset \SS$

%\prob{P3X}  define eigenspace $W_\lambda$ for $A$; assume $A,B$ commute; show $W_\lambda$ is invariant under$B$; show that if $A$ is diagonalized by an ON basis and if each eigenvalue of $A$ has multiplicity one then $B$ is also diagonalized




\end{document}
