\documentclass[12pt]{amsart}
%prepared in AMSLaTeX, under LaTeX2e
\addtolength{\oddsidemargin}{-.55in} 
\addtolength{\evensidemargin}{-.55in}
\addtolength{\topmargin}{-.4in}
\addtolength{\textwidth}{1.1in}
\addtolength{\textheight}{.6in}

\renewcommand{\baselinestretch}{1.05}

\usepackage{verbatim} % for "comment" environment

\usepackage{palatino}

\newtheorem*{thm}{Theorem}
\newtheorem*{defn}{Definition}
\newtheorem*{example}{Example}
\newtheorem*{problem}{Problem}
\newtheorem*{remark}{Remark}

\usepackage{fancyvrb,xspace}

\usepackage[final]{graphicx}

% macros
\usepackage{amssymb}

\usepackage{hyperref}
\hypersetup{pdfauthor={Ed Bueler},
            pdfcreator={pdflatex},
            colorlinks=true,
            citecolor=blue,
            linkcolor=red,
            urlcolor=blue,
            }

\newcommand{\br}{\mathbf{r}}
\newcommand{\bv}{\mathbf{v}}
\newcommand{\bx}{\mathbf{x}}
\newcommand{\by}{\mathbf{y}}

\newcommand{\cH}{\mathcal{H}}
\newcommand{\cL}{\mathcal{L}}
\newcommand{\cV}{\mathcal{V}}
\newcommand{\cW}{\mathcal{W}}

\newcommand{\CC}{\mathbb{C}}
\newcommand{\NN}{\mathbb{N}}
\newcommand{\RR}{\mathbb{R}}
\newcommand{\ZZ}{\mathbb{Z}}

\newcommand{\eps}{\epsilon}
\newcommand{\grad}{\nabla}
\newcommand{\lam}{\lambda}
\newcommand{\lap}{\triangle}

\newcommand{\ip}[2]{\ensuremath{\left<#1,#2\right>}}

\newcommand{\image}{\operatorname{im}}
\newcommand{\onull}{\operatorname{null}}
\newcommand{\rank}{\operatorname{rank}}
\newcommand{\range}{\operatorname{range}}
\newcommand{\trace}{\operatorname{tr}}

\newcommand{\prob}[1]{\bigskip\noindent\textbf{#1.}\quad }
\newcommand{\exer}[2]{\prob{Exercise #2 in Lecture #1}}

\newcommand{\pts}[1]{(\emph{#1 pts}) }
\newcommand{\epart}[1]{\medskip\noindent\textbf{(#1)}\quad }
\newcommand{\ppart}[1]{\,\textbf{(#1)}\quad }

\newcommand{\Matlab}{\textsc{Matlab}\xspace}
\newcommand{\Octave}{\textsc{Octave}\xspace}
\newcommand{\Python}{\textsc{Python}\xspace}
\newcommand{\Julia}{\textsc{Julia}\xspace}

\newcommand{\fl}{\operatorname{fl}}

\newcommand{\ds}{\displaystyle}

\DefineVerbatimEnvironment{mVerb}{Verbatim}{numbersep=2mm,
frame=lines,framerule=0.1mm,framesep=2mm,xleftmargin=4mm,fontsize=\footnotesize}

\newcommand{\nex}{\medskip\noindent}


\begin{document}
\scriptsize \noindent Math 617 Functional Analysis (Bueler) \hfill \emph{revised; assigned 26 January 2024}
\normalsize\medskip

\Large\centerline{\textbf{Assignment 2}}
\large
\medskip

\centerline{\textbf{Due Friday 2 February 2024, at the start of class}}
\medskip
\normalsize

\thispagestyle{empty}

\bigskip
\noindent This Assignment is based primarily sections 2.3, 2.4, 2.6, and 2.7 of our textbook,\footnote{D.~Borthwick (2020).~\emph{Spectral Theory: Basic Concepts and Applications}, Springer} but see also sections 2.1 and 2.2, and the handout.

\medskip
\noindent \textsc{Do the following exercises.}
\smallskip

\prob{P8}  \emph{This is Exercise 2.1.  Note that ``bounded'' is defined on page 9 and ``continuous'' was defined on the handout.}

\nex For normed vector spaces $\cV$ and $\cW$, prove that a linear map $T:\cV\to\cW$ is bounded if and only if it is continuous.


\prob{P9}  \emph{This is Exercise 2.5.  Note that $\|T\|$ is defined on page 9.}

\nex For $T\in\cL(\cH)$, prove that
	$$\|T\| = \sup_{v,w\ne 0} \frac{|\ip{v}{Tw}|}{\|v\|\,\|w\|}.$$


\prob{P10}  \emph{This is Exercise 2.7.  Weak convergence of a sequence in $\cH$ is defined on page 27.  You may use Corollary 2.36.}

\nex Let $\cH$ be a Hilbert space and suppose $\{e_n\}_{n\in\NN}$ is an orthonormal set.  Prove that the sequence $(e_n)$ converges weakly to 0. 


\prob{P11}  Prove directly, without using the Heine-Borel theorem, that the set
	$$K = \{0\} \cup \left\{\tfrac{1}{n}\,:\,n \in \NN\right\} \subset \RR$$
is compact in the usual topology on $\RR$.


\clearpage \newpage
\prob{P12}  \emph{This example is so important that I did it in class \emph{and} I want you to write out the details!  You may use, without comment, the standard properties of integration, as they apply to functions in $L^1(0,1)$.}

\epart{a}  Consider the Banach space $\cV = L^1(0,1)$ and the linear operator
	$$(Af)(x) = \int_0^x f(t)\,dt$$
for $f\in\cV$.  Show that $Af \in \cV$.  Also show $A$ is bounded.

\nex \emph{By part \emph{\textbf{(a)}}, we may write $A\in\cL(\cV)$.}

\medskip
\epart{b}  Show that, in fact, if $f\in \cV$ then $Af$ is a continuous function on $[0,1]$.  (\emph{Show this directly, even though it is also stated in the handout as a fact.  You may use result (A.6) in Appendix A, which is nearly stating what you are trying to prove.})  Observe that $(Af)(0)=0$.

\nex \emph{In the next part, you may use, without comment, the form of the Fundamental Theorem of Calculus in the handout.  You may also use the fact that the only continuous functions $y(x)$ satisfying $y'(x) = \alpha y(x)$, for $\alpha\in\CC$, on $x$ in any non-trivial interval of the real line, are the functions $y(x) = c e^{\alpha x}$ for $c\in\CC$.}

\medskip

\epart{c}  By definition, $f \in \cV$ is an \emph{eigenfunction} of $A$ if $f\ne 0$ and there is $\lambda \in \CC$ so that $Af=\lambda f$.\footnote{Pay attention here. ``$f\ne 0$'' means $f$ is not the zero vector of $\cV$.  Which means what about the pointwise values $f(x)$?  Also, ``$Af=\lambda f$`` means what?  (Think about \emph{almost everywhere}.)}  If $f$ is an eigenfunction of $A$ then we call the corresponding $\lambda$ the \emph{eigenvalue}.  Show that $A$ has no eigenvalues.

\end{document}
