\documentclass[12pt]{amsart}
%prepared in AMSLaTeX, under LaTeX2e
\addtolength{\oddsidemargin}{-.55in} 
\addtolength{\evensidemargin}{-.55in}
\addtolength{\topmargin}{-.4in}
\addtolength{\textwidth}{1.1in}
\addtolength{\textheight}{.6in}

\renewcommand{\baselinestretch}{1.05}

\usepackage{verbatim} % for "comment" environment

\usepackage{palatino}

\newtheorem*{thm}{Theorem}
\newtheorem*{defn}{Definition}
\newtheorem*{example}{Example}
\newtheorem*{problem}{Problem}
\newtheorem*{remark}{Remark}

\usepackage{fancyvrb,xspace,dsfont}

\usepackage[final]{graphicx}

% macros
\usepackage{amssymb}

\usepackage{hyperref}
\hypersetup{pdfauthor={Ed Bueler},
            pdfcreator={pdflatex},
            colorlinks=true,
            citecolor=blue,
            linkcolor=red,
            urlcolor=blue,
            }

\newcommand{\br}{\mathbf{r}}
\newcommand{\bv}{\mathbf{v}}
\newcommand{\bx}{\mathbf{x}}
\newcommand{\by}{\mathbf{y}}

\newcommand{\cD}{\mathcal{D}}
\newcommand{\cF}{\mathcal{F}}
\newcommand{\cH}{\mathcal{H}}
\newcommand{\cL}{\mathcal{L}}
\newcommand{\cV}{\mathcal{V}}
\newcommand{\cW}{\mathcal{W}}

\newcommand{\CC}{\mathbb{C}}
\newcommand{\NN}{\mathbb{N}}
\newcommand{\RR}{\mathbb{R}}
\newcommand{\ZZ}{\mathbb{Z}}

\newcommand{\eps}{\epsilon}
\newcommand{\grad}{\nabla}
\newcommand{\lam}{\lambda}
\newcommand{\lap}{\triangle}

\newcommand{\ip}[2]{\ensuremath{\left<#1,#2\right>}}

\newcommand{\image}{\operatorname{im}}
\newcommand{\onull}{\operatorname{null}}
\newcommand{\rank}{\operatorname{rank}}
\newcommand{\range}{\operatorname{range}}
\newcommand{\trace}{\operatorname{tr}}

\newcommand{\Span}{\operatorname{span}}

\newcommand{\prob}[1]{\bigskip\noindent\textbf{#1.}\quad }
\newcommand{\exer}[2]{\prob{Exercise #2 in Lecture #1}}

\newcommand{\pts}[1]{(\emph{#1 pts}) }
\newcommand{\epart}[1]{\medskip\noindent\textbf{(#1)}\quad }
\newcommand{\ppart}[1]{\,\textbf{(#1)}\quad }

\newcommand{\Matlab}{\textsc{Matlab}\xspace}
\newcommand{\Octave}{\textsc{Octave}\xspace}
\newcommand{\Python}{\textsc{Python}\xspace}
\newcommand{\Julia}{\textsc{Julia}\xspace}

\newcommand{\fl}{\operatorname{fl}}

\newcommand{\ds}{\displaystyle}

\DefineVerbatimEnvironment{mVerb}{Verbatim}{numbersep=2mm,
frame=lines,framerule=0.1mm,framesep=2mm,xleftmargin=4mm,fontsize=\footnotesize}

\newcommand{\nex}{\medskip\noindent}


\begin{document}
\scriptsize \noindent Math 617 Functional Analysis (Bueler) \hfill \emph{assigned 29 March 2024}
\normalsize\medskip

\Large\centerline{\textbf{Assignment 6}}
\large
\medskip

\centerline{\textbf{Due FIXME 2024}}
\medskip
\normalsize

\thispagestyle{empty}

\bigskip
\noindent This Assignment is based primarily on sections FIXME of our textbook, Borthwick (2020)~\emph{Spectral Theory: Basic Concepts and Applications}, Springer.

\medskip
\noindent \textsc{Please do the following exercises.}
\smallskip

\renewcommand{\SS}{\mathbb{S}}

\prob{P28}  a) show that unitary $U$ is bounded; thus from now on once you prove $\ip{Ux}{Uy}=\ip{x}{y}$ for all $x,y$ in domain of operator $U$ then can assume domain is $\cH$

b) show that if $U$ has eigenvalue then it is on unit circle $\SS$

c) $U=M_f$ on $L^2(0,1)$ where $f(x) = e^{2\pi i x}$; show unitary, but show it has not eigenvalues; show approximate eigenvalues


\prob{P29}  a) define $M_{x^2}$ and $T=d^2/dx^2$ on $L^2(\RR)$, with domains so that self-adjoint; show they have no eigenvalues

b) $M_{x^2}$ and $T=d^2/dx^2$ on $L^2(-\pi,\pi)$, with domains so that self-adjoint; show $M$ has no eigenvalues but $T$ does; show $M$ bounded


\prob{P30}  show $\sigma(U) \subset \SS$  [ON A7?]

\prob{P31} $T\in\cL(\cH)$ implies $\sigma(T) \subset B_{\|T\|}(0)$ [?]

\prob{P32} positive $\implies$ symmetric [?]

\end{document}
