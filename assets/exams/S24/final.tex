\documentclass[12pt]{amsart}
%\pagestyle{empty} 
\setlength{\topmargin}{-0.5in} % usually -0.25in
\addtolength{\textheight}{1.2in} % usually 1.25in
\addtolength{\oddsidemargin}{-0.95in}
\addtolength{\evensidemargin}{-0.95in}
\addtolength{\textwidth}{1.9in} %\setlength{\parindent}{0pt}

\newcommand{\normalspacing}{\renewcommand{\baselinestretch}{1.1}\tiny\normalsize}
\normalspacing

% macros
\usepackage{amssymb,xspace}
\usepackage[final]{graphicx}
\usepackage[pdftex,colorlinks=true]{hyperref}
\usepackage{fancyvrb}
\usepackage{tikz}

\newtheorem*{lem*}{Lemma}

\newcommand{\bb}{\mathbf{b}}
\newcommand{\bc}{\mathbf{c}}
\newcommand{\bbf}{\mathbf{f}}
\newcommand{\bs}{\mathbf{s}}
\newcommand{\bu}{\mathbf{u}}
\newcommand{\bv}{\mathbf{v}}
\newcommand{\bx}{\mathbf{x}}
\newcommand{\by}{\mathbf{y}}

\newcommand{\cA}{\mathcal{A}}
\newcommand{\cD}{\mathcal{D}}
\newcommand{\cH}{\mathcal{H}}
\newcommand{\cL}{\mathcal{L}}
\newcommand{\cV}{\mathcal{V}}

\newcommand{\CC}{{\mathbb{C}}}
\newcommand{\RR}{{\mathbb{R}}}
\newcommand{\eps}{\epsilon}
\newcommand{\ZZ}{{\mathbb{Z}}}
\newcommand{\ZZn}{{\mathbb{Z}}_n}
\newcommand{\NN}{{\mathbb{N}}}
\newcommand{\ip}[2]{\left<#1,#2\right>}

\renewcommand{\Re}{\operatorname{Re}}
\renewcommand{\Im}{\operatorname{Im}}
\newcommand{\Log}{\operatorname{Log}}

\newcommand{\range}{\operatorname{range}}

\newcommand{\grad}{\nabla}

\newcommand{\ds}{\displaystyle}

\newcommand{\prob}[1]{\bigskip\noindent\textbf{#1.} }
\newcommand{\pts}[1]{(\emph{#1 pts})}

\newcommand{\probpts}[2]{\prob{#1} \pts{#2} \quad}
\newcommand{\ppartpts}[2]{\textbf{(#1)} \pts{#2} \quad}
\newcommand{\epartpts}[2]{\medskip\noindent \textbf{(#1)} \pts{#2} \quad}


\begin{document}
\hfill \Large Name:\underline{\phantom{Ed Bueler really really long long long name}}
\medskip

\scriptsize \noindent Math 617 Functional Analysis (Bueler) \hfill Friday 3 May 2024
\medskip

\LARGE\centerline{\textbf{Final Quiz}}

\normalsize
\smallskip
\begin{quote}
%\large
\textbf{In-class or proctored.  No book, notes, electronics, calculator, internet access, or communication with other people.  \underline{Precise} statements of definitions, theorems, and lemmas are expected.  100 points possible. \,\underline{65 minutes} maximum.}
\end{quote}

\normalsize
\medskip

\thispagestyle{empty}

\probpts{1}{5}  Define a \textbf{complex Hilbert space}.
\vfill

\probpts{2}{5}  Define a \textbf{unitary map}.
\vfill

\probpts{3}{10}  For an operator $T$ on a complex Hilbert space, whether bounded or unbounded, define the \textbf{spectrum} $\sigma(T)$ and the \textbf{resolvent set} $\rho(T)$. 
\vspace{3.5in}

\clearpage\newpage
\probpts{4}{15}  State the \textbf{Riesz lemma}, which describes the dual space of a (complex) Hilbert space.
\vspace{3.0in}

\probpts{5}{10}  For a (generally) unbounded operator $T$ on a complex Hilbert space, define the \textbf{domain $\cD(T^*)$ of the adjoint}, and the action of the \textbf{adjoint $T^*$}.
\vfill

\clearpage\newpage
\probpts{6}{10}  For a (generally) unbounded operator on a complex Hilbert space, defined what it means to be \textbf{symmetric}, and what it means to be \textbf{self-adjoint}.
\vfill

\probpts{7}{15}  State the \textbf{Riesz (or Riesz-Markov-Kakutani) representation theorem}.
\vfill

\clearpage\newpage
\probpts{8}{15}  State the \textbf{multiplication-operator form of the spectral theorem for self-adjoint operators}.
\vfill

\clearpage\newpage
\probpts{9}{15}  State the \textbf{bounded Borel functions} form of the \textbf{functional calculus for self-adjoint operators}.

%\hrulefill
\clearpage
\newpage
\thispagestyle{empty}
\begin{center}
\small
\textsc{blank space (full page)}
\end{center}
\vfill

\end{document}
