\documentclass[12pt]{article}
\usepackage[top=1.2in, bottom=0.8in, left=1.0in, right=1.2in]{geometry}

\usepackage{graphicx,color,enumitem,fancyvrb}
\usepackage{amsmath,amsthm,amsbsy,amssymb}
\usepackage{palatino}
\usepackage{mdframed}

\usepackage{tikz}
\usepackage[colorlinks=true]{hyperref}

\makeatletter

%% The following commands put defined left and right headers on the top, and a page number
%% on the bottom of all pages beyond page 1
\usepackage{fancyhdr}
\pagestyle{fancy}
\fancyfoot[C]{\ifnum \value{page} > 1\relax\thepage\fi}
\fancyhead[L]{\ifx\@doclabel\@empty\else\@doclabel\fi}
\fancyhead[R]{\ifx\@docdate\@empty\else\@docdate\fi}
\headheight 15pt
\def\doclabel#1{\gdef\@doclabel{#1}}
\def\docdate#1{\gdef\@docdate{#1}}
\makeatother

%% General formatting parameters
\parindent 0pt
\parskip 6pt plus 1pt

\newcommand{\bA}{\mathbf{A}}
\newcommand{\bC}{\mathbf{C}}
\newcommand{\bI}{\mathbf{I}}
\newcommand{\bX}{\mathbf{X}}

\newcommand{\cV}{\mathcal{V}}

\newcommand{\CC}{\mathbb{C}}
\newcommand{\RR}{\mathbb{R}}

%\newcommand{\exer}[1]{\noindent \textbf{Problem #1.} \,}
%\newcommand{\epart}[1]{\noindent \textbf{(#1)} \,}
\newcommand{\eps}{\epsilon}
\newcommand{\ds}{\displaystyle}

%\newcommand{\sect}[1]{\medskip\noindent \textbf{#1.}}
\newcommand{\sect}[1]{\subsection*{#1.}}

\newcommand{\defin}{\emph{Definition.}\,\,}
\newcommand{\thm}{\emph{Theorem.}\,\,}

\newlist{enumex}{enumerate}{3}
\setlist[enumex]{label={Example \Alph*.},leftmargin=27mm,before=\raggedright}

\newcommand{\exer}[2]{\emph{\underline{Exercise.}\, #2} \vspace*{#1mm}}
\newcommand{\instruct}[2]{\emph{\underline{Instructions:}\, #2} \vspace*{#1mm}}
\newcommand{\showit}[2]{\emph{\underline{Show it!} (#2)} \vspace*{#1mm}}



\doclabel{Math 617 Functional Analysis}
\docdate{January 2024; Bueler}

\begin{document}
\strut
\centerline{{\Large \textbf{Handout: Definitions and facts}}}

\centerline{{\Large\strut \textbf{(which you will need to get started)}}}
\bigskip

Functional analysis is the study of vector spaces which have a topology.  Therefore you need to have some sense of what a ``topology'' is.  In fact, our textbook (D.~Borthwick, \emph{Spectral Theory: Basic Concepts and Applications}, GTM 284, Springer, 2020) assumes that you know the basics of vector spaces, topology, measures, and integrals.  This handout is an attempt to get you up to speed, so that you can read the book with real understanding.  When you do read, please stop and ask yourself ``do I understand the definition of this term?''


\sect{Metric spaces} \label{topic:metric}

To talk about topology we start with a ``metric'' and then define open and closed sets from that.  A metric is a generalized distance function.

\defin Suppose $X$ is any set (of ``points'').  A function $d:X\times X\to \RR$ is a \textbf{metric} if, for all $x,y,z\in X$, these conditions hold:
\begin{enumerate}
\item $d(x,y)\ge 0$, and $d(x,y)=0$ if and only if $x=y$
\item $d(x,y)=d(y,x)$ \hfill (symmetry)
\item $d(x,z) \le d(x,y) + d(y,z)$ \hfill (triangle inequality)
\end{enumerate}

The addition and inequality symbols in condition 3 acts on real numbers, and not on the elements of the general set $X$.  (We may not be able to add elements of $X$!)  Also observe that one cannot substitute $\CC$ for $\RR$ in this definition because an ordering ``$\le$'' is not available for $\CC$.  If you have seen ``norms'' on vector spaces then the above definition will ring a bell; see page \pageref{topic:norms}.

\defin If $X$ is a set and $d$ is a metric then one calls the pair $(X,d)$ a \textbf{metric space}.

Here are two examples of metric spaces:

\begin{enumex}
\item Let $X=S^1$ be the unit sphere, namely the set of points in $\RR^n$ which are Euclidean distance 1 from the origin.  Let $d$ be the Euclidean distance between points of $S^1$.

\showit{20}{Argue that $d$ defines a metric.}

\clearpage
\vspace*{20mm}

\item Let $X$ be any set whatsoever.  Define $d(x,x)=0$ and $d(x,y)=1$ if $x\ne y$.

\showit{50}{Argue that $d$ defines a metric.  Note $X$ could be very general, such as the set of all words in the English language.}
\end{enumex}


\sect{Vector spaces}  Many metric spaces are actually vector spaces with norms.  Because you should already know what an abstract vector space is, the definition is an exercise.  Then we get to norms below.

\instruct{0}{Fill in the axioms for a vector space, to complete the definition below.  Add bullet points as needed.}

\defin A set $\cV$ (of vectors) with an operation $*:\CC\times \cV \to \cV$ and another operation $+:\cV\times \cV\to \cV$ is a \textbf{(complex) vector space} if the following hypotheses and conditions hold:
\begin{itemize}
\item \phantom{x} \vspace{5mm}

\item \phantom{x} \vspace{5mm}

\item \phantom{x} \vspace{5mm}

\item \phantom{x} \vspace{5mm}

\item \phantom{x} \vspace{5mm}

\item \phantom{x} \vspace{5mm}

\item \phantom{x} \vspace{25mm}

\end{itemize}


\sect{Norms} \label{topic:norms}

Section 2.1 of the textbook defines a ``norm''.

\instruct{0}{Fill in the definition below.  Refer to Section 2.1 as needed.}

\defin A \textbf{norm} on a complex vector space $\cV$ is a function $\|\cdot\|:\cV \to \RR$ satisfying, for all $v,w\in\cV$ and $a\in \CC$,
\begin{enumerate}
\item \phantom{foo} \vspace{5mm}

\item \phantom{foo} \vspace{5mm}

\item \phantom{foo} \vspace{7mm}

\end{enumerate}

\defin If $\cV$ is a vector space and $\|\cdot\|$ is a norm on it then we say $(\cV,\|\cdot\|)$ is a \textbf{normed vector space}.

Every normed vector space is also a metric space, which the next Exercise asks you to show.  While a norm is similar to a metric, norms require more structure.  You need to be able to add the elements of the set itself, and multiply them by scalars.  That is, the elements of the set have to be vectors, not just ``points''.

\exer{50}{Show that if $\|\cdot\|$ is a norm on $\cV$ then the distance function defined in Section 2.1, namely $\operatorname{dist}(v,w) := \|v-w\|$, is a metric, so $(\cV,\operatorname{dist})$ is a metric space.}


\sect{Open and closed sets}

Once you have a metric (or a norm) then you can define ``open'' and ``closed'' subsets, and talk about the ``boundaries'' of sets.  The starting point is to define a ``ball'' around a point.

\defin Suppose $(X,d)$ is a metric space.  Suppose $x\in X$ is a point and $\eps>0$ is a real number.  The \textbf{open ball} of radius $\eps>0$ around $x$ is the set
	$$B_\eps(x) = \left\{y\in X\,:\,d(x,y) < \eps\right\}$$ 

\defin Suppose $(X,d)$ is a metric space.
\begin{itemize}
\item A subset $Y\subset X$ is \textbf{open} if for all $y\in Y$ there is $\eps>0$ so that $B_\eps(y) \subset Y$.
\item A subset $Y\subset X$ is \textbf{closed} if $X \setminus Y$ is open.
\end{itemize}

\exer{60}{Suppose $(\cV,\|\cdot\|)$ is a normed vector space.  Rewrite the above definitions using the norm.}

\exer{60}{Sketch an open set in the plane $\RR^2$, and illustrate the definition of open set.}

\clearpage\newpage
\exer{60}{Generally speaking, many sets are neither open nor closed.  Sketch such an example of such a set $Y$ in the plane $\RR^2$.}

\defin Suppose $(X,d)$ is a metric space and $Y \subset X$ is any subset.  A point $y\in X$ is in the \textbf{boundary of } $Y$ if for all $\eps>0$, the intersection $B_\eps(y)\cap Y$ is non-empty and the intersection $B_\eps(y)\cap (X\setminus Y)$ is also non-empty.  We write $\partial Y$ for the set of points which are in the boundary of $Y$.

\exer{1}{Using a different color, add $\partial Y$ to your above example.}

Finally, here is a promised definition.

\defin Suppose $X$ is a set, and if $\{Y_\alpha\}$ is a collection of subsets of $X$ which we call the ``open sets'', then we say $\{Y_\alpha\}$ is a \textbf{topology} on $X$.

In fact, the above definition is too informal.  The collection of subsets must satisfy certain requirements to qualify as a system of open sets.  The collection must include the empty set $\emptyset$ and the whole set $X$, it must be closed under arbitrary unions, and the intersection of any pair of subsets in the collection must be in the collection.

From this definition we can make sense of a phrase like ``functional analysis is the study of vector spaces which have a topology.''  Having a system of identified open sets makes it possible to talk about limits, continuity, and so on, and ultimately to find solutions to hard problems.


\sect{Limits and continuity}

From the definition of open sets, or more directly by using open balls, we can define the limit of a sequence.

\defin Suppose $(X,d)$ is a metric space and $(x_n)$ is a sequence of points from $X$.  We say that \textbf{the limit of $(x_n)$ is $\hat x$}, written
	$$\lim_{n\to\infty} x_n = \hat x, \qquad \text{or} \qquad x_n \to \hat x,$$
if
\begin{enumerate}
\item for each open set $Y$ which contains $\hat x$ there is $N$ so that $n\ge N$ implies $x_n \in Y$, or
\item for each $\eps>0$ there is $N$ so that $n\ge N$ implies $x_n \in B_\eps(\hat x)$.
\end{enumerate}

\exer{40}{Show that the two definitions are equivalent.}

\exer{30}{Write a third equivalent form of the definition using only the metric $d$.  (Do not mention open sets or $B_\eps(\cdot)$.)}

\defin Suppose $(X,d)$ is a metric space and that $\hat x = \lim_{n\to\infty} x_n$ for some sequence $(x_n)$.  We say that the sequence \textbf{converges}.

\exer{40}{Show that a sequence cannot converge to two different limits.}

Now consider real-valued functions.

\defin Suppose $(X,d)$ is a metric space and $f:X\to \RR$ is a function.  We say that $f$ is \textbf{continuous at $\hat x\in X$} if for all sequences $(x_n)$ such that $\hat x = \lim_{n\to\infty} x_n$, it holds that
    $$\lim_{n\to\infty} f(x_n) = f(\hat x).$$
We say that $f$ is \textbf{continuous} if it is continuous at every $\hat x\in X$.

\exer{40}{Let $f(x)=x^2$ for $x\in \RR$.  Show that $f$ is continous.}

%\clearpage\newpage
\exer{40}{(This follows Example B on page \pageref{topic:metric}.  It is an extreme case.)  Let $X$ be any set and define $d(x,x)=0$ and $d(x,y)=1$ if $x\ne y$.  We have already shown $d$ is a metric.  Let $f:X\to\RR$ be \emph{any} function.  Show that $f$ is continuous.}


\sect{Cauchy sequences and completeness}

When we have a topology on $X$ then we can discuss whether a given sequence $(x_n)$ of points converges; see the above definition.  However, that definition requires that we ``know'' which is the limit $\hat x \in X$.  That is, convergence of a sequence means convergence to a particular limit.

In certain metric space topologies this becomes easier because there is an easier-to-check condition that is equivalent to convergence to some limit.  The condition only compares elements of the sequence.

\defin Suppose $(X,d)$ is a metric space and $(x_n)$ is a sequence.  We say that $(x_n)$ is a \textbf{Cauchy sequence} if for any $\eps>0$ there is $N$ so that if $m \ge N$ and $n\ge N$ then $d(x_m,x_n)<\eps$.

\exer{50}{Rewrite the above definition for a normed vector space.  Compare the resulting definition to the one stated in section 2.1 of our textbook.  Show that the two definitions are the same.}

\exer{40}{Show that if $(x_n)$ converges to limit $\hat x$ then $(x_n)$ is Cauchy.}

\defin Suppose $(X,d)$ is a metric space.  If every Cauchy sequence $(x_n)$ in $X$ has a limit (there is $\hat x\in X$ so that $\lim_{n\to\infty} x_n=\hat x$) then we say that $(X,d)$ is \textbf{complete}.

We will accept the following Theorem without proof.

\thm The real numbers $\RR$, with the usual metric, is a complete metric space.  Likewise the complex numbers $\CC$ is a complete metric space.

\exer{50}{Let $x_0=2$ and define $x_{k+1}$ as the real zero of the tangent line to $y=x^2-2$ at $x_k$.  Argue geometrically, from the graph of $y=x^2-2$, that $(x_k)$ is Cauchy.  Note that $x_k$ is always a rational number.  What is the limit of $(x_k)$?}


Furthermore, we are now at the start of functional analysis!  The following definition is the main idea of section 2.1 of our textbook.

\defin Suppose $(\cV,\|\cdot\|)$ is a normed vector space, and that it is complete.  We call $(\cV,\|\cdot\|)$ a \textbf{Banach space}.



\sect{Compact sets}

foo

\thm (Extreme value theorem) bar



\sect{Linear map}

\sect{Linear independence}

\sect{Measure}

\sect{Integral}

\end{document}
