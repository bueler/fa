\documentclass[12pt]{article}
\usepackage[top=1.2in, bottom=0.8in, left=1.0in, right=1.2in]{geometry}

\usepackage{graphicx,color,enumitem,fancyvrb}
\usepackage{amsmath,amsthm,amsbsy,amssymb,bm}
\usepackage{palatino}
\usepackage{mdframed}

\usepackage{tikz}
\usepackage[colorlinks=true]{hyperref}

\makeatletter

%% The following commands put defined left and right headers on the top, and a page number
%% on the bottom of all pages beyond page 1
\usepackage{fancyhdr}
\pagestyle{fancy}
\fancyfoot[C]{\ifnum \value{page} > 1\relax\thepage\fi}
\fancyhead[L]{\ifx\@doclabel\@empty\else\@doclabel\fi}
\fancyhead[R]{\ifx\@docdate\@empty\else\@docdate\fi}
\headheight 15pt
\def\doclabel#1{\gdef\@doclabel{#1}}
\def\docdate#1{\gdef\@docdate{#1}}
\makeatother

%% General formatting parameters
\parindent 0pt
\parskip 6pt plus 1pt

\newcommand{\bA}{\mathbf{A}}
\newcommand{\bC}{\mathbf{C}}
\newcommand{\bI}{\mathbf{I}}
\newcommand{\bX}{\mathbf{X}}

\newcommand{\bzero}{\bm{0}}

\newcommand{\cC}{\mathcal{C}}
\newcommand{\cM}{\mathcal{M}}
\newcommand{\cS}{\mathcal{S}}
\newcommand{\cV}{\mathcal{V}}
\newcommand{\cW}{\mathcal{W}}

\newcommand{\CC}{\mathbb{C}}
\newcommand{\QQ}{\mathbb{Q}}
\newcommand{\RR}{\mathbb{R}}

\newcommand{\range}{\operatorname{range}}
\newcommand{\Span}{\operatorname{span}}

%\newcommand{\exer}[1]{\noindent \textbf{Problem #1.} \,}
%\newcommand{\epart}[1]{\noindent \textbf{(#1)} \,}
\newcommand{\eps}{\epsilon}
\newcommand{\ds}{\displaystyle}

%\newcommand{\sect}[1]{\medskip\noindent \textbf{#1.}}
\newcommand{\sect}[1]{\subsection*{#1.}}

\newcommand{\defin}{\emph{Definition.}\,\,}
\newcommand{\lem}{\emph{Lemma.}\,\,}
\newcommand{\thm}{\emph{Theorem.}\,\,}

\newlist{enumex}{enumerate}{3}
\setlist[enumex]{label={Example \Alph*.},leftmargin=27mm,before=\raggedright}

\newcommand{\exer}[2]{\emph{\underline{Exercise.}\, #2} \vspace*{#1mm}}
\newcommand{\instruct}[2]{\emph{\underline{Instructions:}\, #2} \vspace*{#1mm}}
\newcommand{\showit}[2]{\emph{\underline{Show it!} (#2)} \vspace*{#1mm}}



\doclabel{Math 617 Functional Analysis}
\docdate{January 2024 (Bueler)}

\begin{document}
\strut
\centerline{{\Large \textbf{Handout: Definitions and facts}}}

\centerline{{\Large\strut \textbf{(which you will need to get started)}}}
\bigskip

Functional analysis is the study of vector spaces which have a topology.  Therefore you need to have some sense of what a ``topology'' is.  In fact, our textbook (D.~Borthwick, \emph{Spectral Theory: Basic Concepts and Applications}, GTM 284, Springer, 2020) assumes that you know the basics of topology, vector spaces, measures, and integrals.  This handout is an attempt to get you up to speed on these topics, so that you can develop real understanding from the book.  Note that good mathematicians regularly stop their reading and ask themselves ``Do I genuinely understand the definition of this term? What little exercises would confirm that understanding?''


\sect{Metric spaces} \label{topic:metric}

We start with a ``metric'', and then define open and closed sets from that, which is a topology.  A metric is a generalized distance function.

\defin Suppose $X$ is any set.  (Elements of $X$ will be called ``points''.)   A function $d:X\times X\to \RR$ is a \textbf{metric} if, for all $x,y,z\in X$, these conditions hold:
\begin{enumerate}
\item $d(x,y)\ge 0$, and $d(x,y)=0$ if and only if $x=y$
\item $d(x,y)=d(y,x)$ \hfill (symmetry)
\item $d(x,z) \le d(x,y) + d(y,z)$ \hfill (triangle inequality)
\end{enumerate}

The addition symbol in condition 3 acts on real numbers, and not on the elements of the general set $X$.  (Addition may not even be meaningful for elements of $X$!)  Also observe that one cannot substitute $\CC$ for $\RR$ in this definition because a reasonable ordering ``$\le$'' is not available for $\CC$.  If you have seen ``norms'' on vector spaces then the above definition will seem familiar; we will get there shortly.

\defin If $X$ is a set and $d$ is a metric then one calls the pair $(X,d)$ a \textbf{metric space}.

Here are two examples of metric spaces:

\begin{enumex}
\item Let $X=S^n$ be the unit sphere, namely the set of points in $\RR^n$ which are Euclidean distance 1 from the origin.  Let $d$ be the Euclidean distance between points of $S^n$.

\showit{20}{Argue that $d$ defines a metric.}

\clearpage
\vspace*{20mm}

\item Let $X$ be any set whatsoever.  Define $d(x,x)=0$ and $d(x,y)=1$ if $x\ne y$.

\showit{50}{Argue that $d$ defines a metric.  Note $X$ could be a very general set.}
\end{enumex}

One should conclude the following general fact from thinking about Example A: every subset of a metric space is also a metric space, with the same metric.


\sect{Vector spaces}  Many metric spaces are actually vector spaces with norms.  We start by defining an abstract vector space, but treated as an exercise.  Do you remember all the conditions?

\instruct{0}{Fill in the axioms for a vector space, to complete the definition below.  Add bullet points as needed.}

\defin A set $\cV$ (of \textbf{vectors}) with a \textbf{scalar multiplication} operation $*:\CC\times \cV \to \cV$ and a \textbf{vector addition} operation $+:\cV\times \cV\to \cV$ is a \textbf{(complex) vector space} if the following hypotheses and conditions hold:
\begin{itemize}
\item \phantom{x} \vspace{5mm}

\item \phantom{x} \vspace{5mm}

\item \phantom{x} \vspace{5mm}

\item \phantom{x} \vspace{5mm}

\item \phantom{x} \vspace{5mm}

\item \phantom{x} \vspace{5mm}

\item \phantom{x} \vspace{25mm}

\end{itemize}


\sect{Norms} \label{topic:norms}

Section 2.1 of the textbook defines a ``norm''.

\instruct{0}{Fill in the definition below.  Refer to Section 2.1 as needed.}

\defin A \textbf{norm} on a complex vector space $\cV$ is a function $\|\cdot\|:\cV \to \RR$ satisfying, for all $v,w\in\cV$ and $a\in \CC$,
\begin{enumerate}
\item \phantom{foo} \vspace{5mm}

\item \phantom{foo} \vspace{5mm}

\item \phantom{foo} \vspace{7mm}

\end{enumerate}

\defin If $\cV$ is a vector space and $\|\cdot\|$ is a norm on it then we say $(\cV,\|\cdot\|)$ is a \textbf{normed vector space}.

Every normed vector space is also a metric space, and the next Exercise asks you to show this.  A norm is thus a special case of a metric; \emph{a norm requires the additional structure of a vector space}.  You need to be able to add elements themselves, and multiply them by scalars; the elements of the set need to be vectors and not just ``points''.

\exer{30}{Show that if $\|\cdot\|$ is a norm on $\cV$ then the distance function defined in Section 2.1, namely $\operatorname{dist}(v,w) := \|v-w\|$, is a metric.  Thus $(\cV,\operatorname{dist})$ is a metric space.}


\sect{Open and closed sets (are topology)}

Once you have a metric on a set, or a norm on a vector space, then you can define ``open'' and ``closed'' subsets, and talk about the ``boundaries'' of sets.  Open sets allow one to talk about ``nearness'' generally.  Our starting point is a ``ball'' around a point.

\defin Suppose $(X,d)$ is a metric space.  If $x\in X$ is a point and $\eps>0$ is a real number then the \textbf{open ball} of radius $\eps>0$ around $x$ is the set
	$$B_\eps(x) = \left\{y\in X\,:\,d(x,y) < \eps\right\}$$ 

\defin Suppose $(X,d)$ is a metric space.
\begin{itemize}
\item A subset $Y\subset X$ is \textbf{open} if for all $y\in Y$ there is $\eps>0$ so that $B_\eps(y) \subset Y$.
\item A subset $Y\subset X$ is \textbf{closed} if $X \setminus Y$ is open.
\end{itemize}

If $(\cV,\|\cdot\|)$ is a normed vector space then the above definitions all make sense, of course, with the formula $B_\eps(x) = \left\{y\in X\,:\,\|x-y\|<\eps\right\}$.

\exer{50}{Sketch an open set in the plane $\RR^2$, and illustrate the definition of open set.}

\exer{50}{Generally speaking, sets are neither open nor closed.  Sketch an example of such a set in the plane $\RR^2$.}

\clearpage\newpage
\defin Suppose $(X,d)$ is a metric space and $Y \subset X$ is any subset.  A point $z\in X$ is in the \textbf{boundary of } $Y$ if for all $\eps>0$, both $B_\eps(z)\cap Y$ and $B_\eps(z)\cap (X\setminus Y)$ are non-empty.  We write $\partial Y$ for the set of points which are in the boundary of $Y$.

\exer{1}{Using a different color, sketch $\partial Y$ onto your above example.}

Finally, here is a promised definition.

\defin Suppose $X$ is a set, and if $\{Y_\alpha\}$ is a collection of subsets of $X$ which are the open sets.  Then we say $\{Y_\alpha\}$ is a \textbf{topology} on $X$.

If $(X,d)$ is a metric space then the above definition suffices because we already know which are the open sets.  However, topology is more general than metric spaces.  On any set a collection of subsets can be defined as the \textbf{open sets} if it satisfies certain requirements.  The collection must include the empty set $\emptyset$ and the whole set $X$, the intersection of any pair of subsets in the collection must be in the collection, and any union of subsets in the collection must be back in the collection.  These conditions all hold if you define open sets via a metric; feel free to treat this as another Exercise!

From the definition of a ``topology'' we can make sense of a phrase like ``functional analysis is the study of vector spaces which have a topology.''  Having a system of identified open sets makes it possible to talk about limits, continuity, and so on, and ultimately to find solutions to hard problems.


\sect{Limits and continuity}

From the definition of open sets, or more directly by using a metric, we can define the limit of a sequence.

\defin Suppose $(X,d)$ is a metric space and $(x_n)$ is a sequence of points from $X$.  We say that \textbf{the limit of $(x_n)$ is $\hat x$}, written
	$$\lim_{n\to\infty} x_n = \hat x, \qquad \text{or} \qquad x_n \to \hat x,$$
if $\hat x \in X$ and if either of the following conditions holds:
\begin{enumerate}
\item for each open set $Y$ which contains $\hat x$ there is $N$ so that $n\ge N$ implies $x_n \in Y$, or
\item for each $\eps>0$ there is $N$ so that $n\ge N$ implies $x_n \in B_\eps(\hat x)$.
\end{enumerate}

\exer{20}{Show that the two definitions are equivalent.}

\clearpage\newpage
\phantom{foo}
\vspace{20mm}

\exer{30}{Write a third equivalent form of the definition using only the metric $d$.  (Do not mention open sets or $B_\eps(\cdot)$.)}

\defin Suppose $(X,d)$ is a metric space and that $\ds \hat x = \lim_{n\to\infty} x_n$ for a sequence $(x_n)$.  We say that the sequence \textbf{converges}.

\exer{40}{Show that a sequence cannot converge to two different limits.}

Now consider complex-valued and real-valued functions.

\defin Suppose $(X,d)$ is a metric space and $f:X\to \CC$ is a function.  We say that $f$ is \textbf{continuous at $\hat x\in X$} if either of the following conditions holds:
\begin{enumerate}
\item for each $\eps>0$ there is $\delta>0$ so that $x \in B_\eps(\hat x)$ implies $|f(x)-f(\hat x)| < \eps$, or
\item for all sequences $(x_n)$ such that $\displaystyle \lim_{n\to\infty} x_n = \hat x$, it holds that $\displaystyle \lim_{n\to\infty} f(x_n) = f(\hat x)$.
\end{enumerate}

The second definition is often called ``sequential continuity''.

\defin We say that $f$ is \textbf{continuous} if it is continuous at every $\hat x\in X$.

\exer{20}{Let $f(x)=x^2$ for $x\in \RR$.  Show that $f$ is continuous.}

\clearpage\newpage
\phantom{foo}
\vspace{20mm}

\exer{40}{(\emph{This follows Example B on page \pageref{topic:metric}.  It is an extreme, entertaining, and not terribly important case.})  Let $X$ be any set and define the metric by $d(x,x)=0$ and $d(x,y)=1$ if $x\ne y$.  Show that if a sequence converges then it is eventually constant.  Then, for \emph{any} function $f:X\to\CC$, show that $f$ is continuous.}


\sect{Cauchy sequences and completeness}

When we have a topology on $X$ and a sequence $(x_n)$ of points in $X$ then the above definition determines whether the sequence converges.  However, the definition requires that we have identified the limit $\hat x \in X$.  That is, convergence of a sequence means convergence to a particular limit.

In certain metric space topologies convergence becomes easier!  That is, sometimes there is an easier-to-check condition that is equivalent to convergence to some limit.  The condition only involves the distance between elements of the sequence.

\defin Suppose $(X,d)$ is a metric space and $(x_n)$ is a sequence of points in $X$.  We say that $(x_n)$ is a \textbf{Cauchy sequence} if for any $\eps>0$ there is $N$ so that if $m \ge N$ and $n\ge N$ then $d(x_m,x_n)<\eps$.

\exer{35}{Show that if a sequence $(x_n)$ converges to limit $\hat x$ then $(x_n)$ is Cauchy.}

\clearpage\newpage
\exer{50}{Rewrite the Cauchy sequence definition for a normed vector space.  Compare the resulting definition to the one stated in section 2.1 of our textbook.  Show that the two definitions are the same.}

\defin Suppose $(X,d)$ is a metric space.  If every Cauchy sequence $(x_n)$ in $X$ has a limit, i.e.~there is $\hat x\in X$ so that $\displaystyle \lim_{n\to\infty} x_n=\hat x$, then we say that $(X,d)$ is \textbf{complete}.

Not all metric spaces are complete.  For example, the rational numbers are a normed vector space, using the usual absolute value as the metric, but there are Cauchy sequences of rational numbers which do not converge.

We will accept the following Theorem without proof.

\thm The real numbers $\RR$, with the usual absolute value norm, is a complete normed vector space, and thus a complete metric space.  Likewise the complex numbers $\CC$ is a complete normed vector space.

\exer{60}{Let $x_0=2$.  For $k=1,2,\dots$ define $x_k$ as the real zero of the tangent line to $y=x^2-2$ at $x_{k-1}$.  Argue that $x_k$ is rational.  Argue geometrically, from the graph of $y=x^2-2$, that $(x_k)$ is Cauchy.  What is the limit of $(x_k)$?}

With the above definition of completeness, we are now at the start of functional analysis!  The following is a major conceptual starting point.

\defin Suppose $(\cV,\|\cdot\|)$ is a normed vector space, and that it is complete metric space.  We call $(\cV,\|\cdot\|)$ a \textbf{Banach space}.


\clearpage\newpage
\sect{Compact sets}

Speaking very informally, a compact subset of a metric space is well-approximated by finitely-many points.  The precise definition seems awkward when you first read it.

\defin Suppose $(X,d)$ is a metric space and $Z \subset X$.  We say that $\{Y_\alpha\}$ is an \textbf{open cover} of $Z$ if each $Y_\alpha$ is open set in $X$ and if
	$$Z \subset \,\bigcup_{\alpha} \,Y_\alpha.$$

\defin We say $K$ is \textbf{compact} if from every open cover $\{Y_\alpha\}$ of $K$ we can choose a finite sub-collection $\{Y_{\alpha_j}\}_{j=1}^n$ which is also an open cover of $K$.

This definition is often said as ``a set is compact if every open cover of $K$ has a finite sub-cover''.

\exer{50}{For $X=\RR^2$, sketch a subset $K$ and sketch the above definitions.}

\exer{40}{Suppose $K\subset X$ is a finite set.  Show that $K$ is compact.}

\exer{35}{Pretend to do numerical stuff as follows: Sketch a grid of balls $B_\eps(x_{j,k})$, an open cover, over the set $K=[0,1]^2$.  This illustrates the informal description of ``compact.''}

\clearpage\newpage
\exer{50}{Suppose $K=[0,1] \subset \RR$.  Sketch the (direct) proof that $K$ is compact.  You will need the fact that $(\RR,|\cdot|)$ is a complete metric space.}

Closed and bounded sets in normed vector spaces are commonly objects of interest in functional analysis.  In finite dimensions these are compact sets.  We will see that they are \emph{not} generally compact in infinite-dimensional normed vector spaces, whether or not those spaces are complete (i.e.~Banach spaces).

We will accept the following important theorem without proof.

\thm (Heine-Borel) If $K \subset \CC^n$ is closed and bounded then $K$ is compact.

An optimization problem for a real-valued function can always be solved if the function is continuous and the input set is compact.  This is the start of the entire field of optimization!

\thm (``Extreme value theorem'') Suppose $(X,d)$ is a metric space, $K\subset X$, and $f:K\to\RR$ (or $f:X\to\RR$) is continuous.  If $K \subset X$ is compact then $f$ attains its maximum and minimum on $K$.

Precisely-stated, the conclusion of this theorem is that there are $p,q\in K$ so that
    $$f(p) \le f(x) \le f(q)$$
for all $x \in K$.



\sect{Linear maps}

\instruct{0}{The core meaning of ``linear'' is another fill-in-the-blanks exercise.}

\defin Suppose $\cV,\cW$ are vector spaces.  A function (map) $L:\cV \to \cW$ is \textbf{linear} if the following two conditions hold:
\begin{enumerate}
\item \phantom{foo}
\item \phantom{foo}
\end{enumerate}

\clearpage\newpage
\exer{60}{Show that if $L:\cV \to \cW$ is linear then $L\bzero = \bzero$.  Also show that the sets
    $$\ker L = \{v\in\cV\,:\,Lv=\bzero\}, \quad \range L = \{w\in\cW\,:\,\exists v \in \cV \text{ so that } Lv=w\}$$
are vector spaces.}

In functional analysis we will have much to say about linear maps, that is, linear maps on vector spaces \emph{which also have a topology}.  The above is what you need to get started.


\sect{Linear combinations}

Sometimes a vector can be built from other vectors in a manner that linear maps will respect, that is, as a linear combination.  However, it is \emph{extremely} important to distinguish between finite sums and infinite sums.  Finite sums are never problematic for linear maps, but infinite sums have issues of convergence, that is, of topology.

\defin Suppose $\cV$ is a complex vector space and $u \in \cV$ is a vector.  Suppose $\cS = \{v_\alpha\}$ is any collection of vectors.  We say $u$ is a \textbf{(finite) linear combination from} $\cS$ if there exist \emph{finitely-many} vectors $\{v_{\alpha_j}\}_{j=1}^n$ from $\cS$, and coefficients $c_j\in\CC$, so that
	$$u = c_1 v_{\alpha_1} + \dots + c_n v_{\alpha_n} = \sum_{j=1}^n c_j v_{\alpha_j}$$

In the above definition $S$ may be a finite or an infinite set.  However, for the purposes of functional analysis we will reserve the words ``linear combination'' for \emph{finite} linear combinations.  We will also use infinite sums of vectors, quite routinely, but these will be called \textbf{series}, the same name for infinite sums as in calculus, and we will pay close attention to whether these series converge.

\clearpage\newpage
\exer{30}{Show that if $L:\cV \to \cW$ is a linear map then \, $\ds L\left(\sum_{j=1}^n c_j v_{\alpha_j}\right) = \sum_{j=1}^n c_j L v_{\alpha_j}$.}

For future reference, note that \emph{any} linear map, whether or not it is continuous, can be applied to finite sums as above.

\defin Suppose $\cV$ is a complex vector space.  Suppose $S = \{v_\alpha\}$ is any set of vectors from $\cV$, whether finite or infinite.  The \textbf{span} of $S$ is the set of all \emph{finite} linear combinations:
	$$\Span S = \left\{\sum_{j=1}^n c_j L v_{\alpha_j}\,:\,c_j \in \CC \text{ and } v_{\alpha_j} \in S\right\}.$$

\defin Suppose $\cV$ is a complex vector space.  If there exists a finite subset $S \subset \cV$ so that $\cV = \Span S$ then we say $\cV$ is \textbf{finite dimensional}.  If not, we say $\cV$ is \textbf{infinite dimensional}.  In the former case $\dim \cV = \min |S|$ is the \textbf{dimension of} $cV$, where the minimum is over $S \subset \cV$ such that $\Span S = \cV$.  (Here $|S|$ denotes the cardinality of $S$.)

\defin Suppose $\cV$ is a complex vector space and $S = \{v_1,\dots,v_n\} \subset \cV$ is a finite subset.  We say $S$ is \textbf{linearly independent} if there exist no coefficients $c_j \in \CC$, other than identically zero coefficients, so that \,$\ds \bzero = \sum_{j=1}^n c_j v_j$.

\exer{30}{Show that the set $\ds \left\{\begin{bmatrix} 1 \\ 0 \\ 0\end{bmatrix}, \begin{bmatrix} 0 \\ 2 \\ 0\end{bmatrix}, \begin{bmatrix} 0 \\ 0 \\ 3 \end{bmatrix}, \begin{bmatrix} 1 \\ 1 \\ 1\end{bmatrix}\right\}$ in $\CC^3$ is linearly dependent.}

\defin Suppose $\cV$ is a complex vector space and that $S \subset \cV$.  If every finite subset of $S$ is linearly independent and if $\Span S = \cV$ then we call $S$ a \textbf{(algebraic or Hamel) basis} for $\cV$.

In the above definition, if $S$ itself is finite then $\cV$ is finite-dimensional and this is the usual definition of ``basis'': a \textbf{basis} is a finite and linearly-independent set of vectors which spans the vector space.

However, if $S$ is infinite, so that $\cV$ is an infinite dimensional vector space, then the above definition of Hamel basis is widely regarded as useless!  This strange situation is despite the fact that the axiom of choice shows that every vector space $\cV$ \emph{has} an algebraic/Hamel basis.  For all of the many parts of mathematics of which I am aware which use infinite-dimensional vector spaces, there are \emph{zero} applications of algebraic/Hamel bases.  In any case the concept never comes up in our textbook.


\sect{Measure}

Our final topic is integration, which is to say \emph{Lebesgue} integration.  We will suppress \emph{many} details and be very brief.  However, Appendix A of our textbook fills in much of the theory, at a more general level than here.  (For even more information, please see a graduate text on real analysis.)  These notes restrict to Lebesgue integration of complex-valued functions over Euclidean space $\RR^n$ (or a subset thereof).

The main preparatory ideas are ``measurable sets'' and (positive) ``measure''.  The following is not really a definition because I am suppressing the precise properties satisfied by measurable subsets of $\RR^n$.  We will accept, without proof, the existence of such a well-defined and useful system of measurable sets.

\defin Consider $\RR^n$ with the usual (Euclidean) norm and metric.  The \textbf{(Lebesgue) measurable subsets} are a collection $\cM$ of subsets of $\RR^n$.  $\cM$ includes all open and closed sets, and furthermore it is closed under countable unions, countable intersections, and complements.\footnote{The technical issue in the construction of $\cM$ is that first you build the Borel measurable sets as the smallest $\sigma$-algebra including the open and closed sets, and then later you build $\cM$ by including all subsets of sets of measure zero.  As I said, I am suppressing such details here.}

Intuitively, if one can describe precisely how a subset of $\RR^n$ is constructed then it will be measurable.  Even ``weird'' sets like fractals and randomly-generated subsets are measurable.  Nonetheless, there are (many) non-measurable subsets, but their ``constructions'' require highly-abstracted steps, e.g.~using the axiom of choice.

\defin A \textbf{(positive) measure} is a function
	$$\mu : \cM \to [0,\infty)$$
with the property of \textbf{countable additivity}, namely that if $A_j \in \cM$ for $j=1,2,\dots$ are disjoint measureable sets then
	$$\mu\left(\bigcup_{j=1}^\infty A_j\right) = \sum_{j=1}^\infty \mu(A_j)$$

Intuitively, a measure reports how big a given set is.  Thus a measure is a generalization of area or volume.  But the concept is very general, and it \emph{also} generalizes the idea of counting the number of points in a set.  In fact, some measures yield large numbers for certain finite sets, whereas other measures give zero for all finite sets; the latter is true of Lebesgue measure.  Again the following definition suppresses many details.

\defin On $\RR^n$, \textbf{Lebesgue measure} is a measure
	$$m : \cM \to [0,\infty)$$
for which the rectangles (products of intervals) have their usual volume:
	$$\text{if }\, R = I_1\times I_2 \times \dots \times I_n \, \text{ then } \, m(R) = \prod_{j=1}^n (b_j - a_j).$$
Here each $I_j$ is an interval which may or may not include its endpoints:
	$$I_j = (a_j,b_j) \,\Big|\, (a_j,b_j] \,\Big|\, [a_j,b_j) \,\Big|\, [a_j,b_j].$$
(Note that all rectangles $R_j$ are themselves Lebesgue measurable.)

A key idea is that the Lebesgue measure $m(A)$ is defined for any set $A\in\cM$, and thus for many subsets of $\RR^n$ which are much more general than products of intervals.  In any case the number $m(A)$ is the $n$-dimensional volume of $A$.  (Some sources denote $m(A)$ by $\operatorname{vol}(A)$.)  Lebesgue measure is well-defined for extraordinarily complicated sets, e.g.~the famous Mandelbrot set in $\RR^2$, but also for any ordinary polygon.

\exer{50}{In any dimension the Lebesgue measure of a single point is zero: $m(\{x\})=0$.  (It follows from the above \dots why?)  Show that the set of rational numbers $\QQ \subset \RR^1$ is measurable and that it is also of measure zero.}

An essential formula, and a key one for understanding Lebesgue measure, is that if $A\in\cM$ then
	$$m(A) = \inf \left\{\sum_{j=1}^\infty m(R_j) \,:\, A \subset \bigcup_{j=1}^\infty R_j\right\}$$
where $R_j$ are products of intervals as above.  That is, for measurable sets $A$ we can compute (imagine computing?) $m(A)$ by covering $A$ with rectangles more and more accurately, and then taking the smallest resulting volume.

\clearpage\newpage
\exer{60}{Argue for why the rule \,$m(T)=\frac{1}{2} b h$\, applies to triangles $T\subset \RR^2$.  For simplicity consider right triangles with vertices at $(0,0)$, $(b,0)$, and $(0,h)$.  Cover $T$ with disjoint rectangles and use the above idea.}

We will accept the following very important fact without proof.

\lem Lebesgue measure is translation invariant: if $A\in \cM$ and $y\in\RR^n$ then $A+y\in\cM$ and
	$$m(A+y) = m(A).$$

%Finally a definition which suggests (correctly) that there are many measures out there.

%\defin  The triple $(\RR^n,\cM,m)$ is called a \textbf{measure space}.  One says that $\RR^n$ is the \textbf{base space}, $\cM$ is the (collection of) \textbf{measurable sets}, and $m$ is the \textbf{measure}.


\sect{Integral}

Some functions are well-behaved with respect to the measurable sets.  For the following definition, recall that $f^{-1}(A)=\{x\in X\,:\,f(x)\in A\}$.

\defin Consider a function $f:X\to\CC$, or $f:X\to \RR$, where the domain is a measurable subset of $\RR^n$: $X \in \cM$.  (Note $X=\RR^n$ itself is a common case.)  The function is \textbf{measurable} if the preimage of every open set is a measurable set:
	$$A \subset \CC \text{ is open} \quad \implies \quad f^{-1}(A) \in \cM.$$


The definition of the integral of a measurable function starts via a simpler class of functions than measurable functions, namely ``simple'' functions.

\defin The \textbf{characteristic function} of a set $E \subset \RR^n$ is
	$$\chi_E(x) = \begin{cases} 1, & x\in E \\ 0, & x \notin E. \end{cases}$$

\defin A function $\varphi:X\to \CC$ is \textbf{simple} if there exist finitely-many measureable sets $E_j \in \cM$, such that $m(E_j)<\infty$, and coefficients $c_j \in \CC$ so that
	$$\boxed{\varphi = \sum_{j=1}^\ell c_j \chi_{E_j}}$$

If $\varphi$ is simple then $|\varphi|$ is also simple.  Note that a simple function attains only finitely-many values.

\exer{55}{Argue that one can always choose the sets $E_j$ to be nonempty and disjoint.  (In which case the coefficients $c_j$ are equal to the attained nonzero values.)}

A key idea about simple functions is that one can approximate more general functions by simple functions based on finely ``cutting up'' the $y$-axis.

\exer{50}{Sketch a continuous function $f:[0,1]\to\RR$.  Now use a different color to sketch a simple function $\varphi$ which approximates it closely ($\varphi\approx f$).  Argue, along the way, that a continuous function is measurable.}

We will accept without proof the following approximation fact.

\lem If $f:X\to \CC$ is measurable then there exists a sequence of simple functions $(\varphi_k)$ such that $\varphi_k \to f$ pointwise, and such that $|\varphi_k|$ increases to $|f|$.

The core idea of Lebesgue integration, versus Riemann integration, is in the following easy definition.

\defin The \textbf{(Lebesgue) integral of a simple function} is the weighted sum of the Lebesgue measures of its constituent sets:
	$$\boxed{\int_X \varphi\,dm = \sum_{j=1}^\ell c_j m(E_j)}.$$

\clearpage\newpage
\exer{50}{Make side-by-side sketches of the Lebesgue integral of a simple function and of some Riemann sums for the same function and integral.}

We can integrate measurable functions via approximation by simple functions.  However, the following definition also reflects the fact that we need to be cautious and avoid ambiguous ``infinity minus infinity'' situations.

\defin Suppose $f:X\to\CC$ is a measurable function.  If $(\varphi_k)$ is a sequence of simple functions as in the previous lemma, and if
    $$\lim_{k\to\infty} \int_X |\varphi_k|\,dm < \infty$$
then we say $f$ is \textbf{integrable}.

For a measurable function $f$, one is free to define the integral of $|f|$ as the above limit whenever the sequence $(\varphi_k)$ comes from the previous lemma, i.e.~$\int_X |f|\,dm = \lim_{k\to\infty} \int_X |\varphi_k|\,dm$, and our textbook does this.  However, the value might be $+\infty$.  The adjective ``integrable'' means that \emph{this limit is finite}.

\defin Suppose $f:X\to\CC$ is an integrable function.  (Note $f$ is, of course, also measurable.)  Then its \textbf{Lebesgue integral} is
    $$\boxed{\int_X f\,dm = \lim_{k\to\infty} \int_X \varphi_k\,dm}$$

We will accept without proof that the integral value is in $\CC$, and that it is unique.  The integral does not depend on \emph{which} approximating sequence $(\varphi_k)$ was used.

Sometimes the integral is written with an independent variable, and it is sometimes not.  The notation ``$dx$'' is (very) common for Lebesgue measure.  In fact, if $X=(a,b)$ is an interval then one can further calculus-ify the notation:
    	$$\int_0^1 f(x)\,dx = \int_X f(x)\,dx = \int_X f\,dm.$$
These are all the same integral.

FIXME def $L^1(a,b)$

To actually compute numbers from integrals, the technique invented by Newton and Leibniz remains the most popular!  FIXME here is one form; Theorem A.7 in the textbook is better but it requires one to define ``absolutely continuous''

\thm (Fundamental Theorem of Calculus)  FIXME in weakest form: if $f$ is $L^1$ then there is a continuous function $F(x)$ with $F'(x)=f(x)$, and $F(b) - F(a) = \int_a^b f(x)\,dx$ 

\exer{60}{(Use several colors for this exercise.)  Sketch $f(x)=1-x^2$ on $[0,1]$.  Sketch simple functions $\varphi_k(x)$ which approximate it from below, and their integrals.  Sketch $\int_0^1 f(x)\,dx$ as an area.  Compute it by the FTC.}

\end{document}
