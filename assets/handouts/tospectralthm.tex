\documentclass[11pt]{article}
\usepackage[top=1.2in, bottom=0.8in, left=1.0in, right=1.2in]{geometry}

\usepackage{graphicx,color,enumitem,verbatim,fancyvrb}
\usepackage{amsmath,amsthm,amsbsy,amssymb,bm}
\usepackage{palatino}
\usepackage{mdframed}

\usepackage{tikz}
\usepackage[colorlinks=true]{hyperref}

%% The following commands put defined left and right headers on the top, and a page number
%% on the bottom of all pages beyond page 1
\usepackage{fancyhdr}
\pagestyle{fancy}
\fancyfoot[C]{\ifnum \value{page} > 1\relax\thepage\fi}
\setlength{\headheight}{15pt}
\fancyhead[L]{{\small Math 617 Functional Analysis}}
\fancyhead[R]{\emph{v1.0} \, {\small 24 March 2024 (Bueler)}}


%% General formatting parameters
\setlength{\parindent}{0pt}
\setlength{\parskip}{6pt plus 1pt}

\newcommand{\bA}{\mathbf{A}}
\newcommand{\bC}{\mathbf{C}}
\newcommand{\bI}{\mathbf{I}}
\newcommand{\bX}{\mathbf{X}}

\newcommand{\bzero}{\bm{0}}

\newcommand{\cC}{\mathcal{C}}
\newcommand{\cD}{\mathcal{D}}
\newcommand{\cH}{\mathcal{H}}
\newcommand{\cL}{\mathcal{L}}
\newcommand{\cM}{\mathcal{M}}
\newcommand{\cS}{\mathcal{S}}
\newcommand{\cV}{\mathcal{V}}
\newcommand{\cW}{\mathcal{W}}

\newcommand{\CC}{\mathbb{C}}
\newcommand{\QQ}{\mathbb{Q}}
\newcommand{\RR}{\mathbb{R}}
\renewcommand{\SS}{\mathbb{S}}

\newcommand{\ip}[2]{\left<#1,#2\right>}

\newcommand{\range}{\operatorname{range}}
\newcommand{\Span}{\operatorname{span}}

\newcommand{\eps}{\epsilon}
\newcommand{\ds}{\displaystyle}

\newcommand{\sect}[1]{\subsection*{#1.}}

\newcommand{\defin}{\emph{Definition.}\,\,}
\newcommand{\lem}{\emph{Lemma.}\,\,}
\newcommand{\thm}{\emph{Theorem.}\,\,}

\newlist{enumex}{enumerate}{3}
\setlist[enumex]{label={Example \Alph*.},leftmargin=27mm,before=\raggedright}


\begin{document}
\strut
\centerline{{\Large \textbf{Definitions and facts leading to the spectral theorem}}}

\normalsize
\medskip

Page numbers are for Borthwick, \emph{Spectral Theory} Springer 2020.

\textbf{Notation:} $\forall$=``for all'', $\exists$=``there exists'', $\cH$ is a separable Hilbert space, $T$ is an (unbounded) operator on $\cH$, $T-z=T-zI$, $U \in \cL(\cH)$ is a unitary operator, and $A$ is an (unbounded) self-adjoint operator on $\cH$.
%\sect{Metric spaces} \label{topic:metric}


\newcommand{\itwo}[2]{{\small \textbf{#1}} {\footnotesize p #2} \,\,}
\newcommand{\df}[1]{\,\itwo{def}{#1}}
\newcommand{\ft}[1]{\itwo{\underline{fact}}{#1}}

\begin{itemize}[leftmargin=10mm,itemsep=0mm]
\item[\df{36}] an \emph{operator} $T$ is a linear map on $\cH$ with a dense domain $\cD(T)$
\item[\df{38}] the \emph{adjoint} of $T$ is an operator $T^*$, with domain
	$$\cD(T^*) = \left\{v\in\cH\,:\,\ell(u)=\ip{v}{Tu} \in \cL(\cH,\CC)\right\},$$
so that $\ip{T^* v}{u} = \ip{v}{Tu}$ \, for all $v\in\cD(T^*)$, $u\in\cD(T)$
\item[\df{41}] an operator is \emph{closed} if its graph is a closed subset of $\cH\times \cH$
\item[\ft{43}] the adjoint $T^*$ is always closed
\item[\ft{44}] $T=T^{**}$ if $T$ is closed
\item[\ft{44}] $T$ closable $\iff$ $\cD(T^*)$ dense
\item[\ft{44}] \textbf{closed graph theorem.} when $\cD(T)=\cH$: \, $T$ closed $\iff$ $T\in\cL(\cH)$
\item[\df{46}] $T$ has \emph{bounded inverse}: \, $\exists$ $T^{-1}\in\cL(\cH)$ s.t.~$TT^{-1}=I$ on $\cH$ and $T^{-1}T=I$ on $\cD(T)$
\item[\ft{46}] $T^{-1}\in\cL(\cH)$ $\iff$ $T$ is closed, $T$ is bounded away from zero, and $\range(T)$ dense
\item[\df{47}] $A$ is \emph{self-adjoint} if $A^*=A$ \hfill {\footnotesize $\leftarrow$ \textbf{requires:} $\cD(A^*)=\cD(A)$}
\item[\df{47}] $T$ is \emph{symmetric} if $\ip{Tu}{v}=\ip{u}{Tv}$ for all $v\in\cD(T)$
\item[\ft{47}] $T$ is symmetric $\implies$ $T$ is closable
\item[\ft{47}] $A$ is self-adjoint $\implies$ $A$ is symmetric
\item[\df{47}] $T$ is \emph{positive} if $\ip{v}{Tv}\ge 0$ for all $v\in\cD(T)$
\item[\df{67}] \emph{eigenvalue} and \emph{eigenvector}: \, $T\phi=\lambda\phi$ \,for $\phi\in\cD(T)\setminus\{0\}$ and $\lambda\in\CC$
\item[\df{68}] \emph{spectrum}: the set $\sigma(T)=\{\lambda \in\CC\,:\,T - \lambda \text{ does not have a bounded inverse}\}$
\item[\df{68}] \emph{resolvent set}: $\rho(T) = \CC\setminus \sigma(T)$
\item[\df{68}] if $z\in\rho(T)$ then $R_z = (T-z)^{-1}$ is the \emph{resolvent} operator
\item[\ft{68}] $\sigma(T)=\CC$ if $T$ is not closed
\item[\ft{69}] $\sigma(T) \subset B_{\|T\|}(0)$ if $T$ is bounded
\item[\ft{69}] $\sigma(T^*) = \sigma(T)^*$, $\rho(T^*)=\rho(T)^*$, and $\left[(T-z)^{-1}\right]^* = \left(T-\overline{z}\right)^{-1}$
\item[\ft{71}] for  $f:X\to\CC$ measurable and $M_f$ a multiplication operator on $L^2(X,d\mu)$:

$\lambda\in\CC$ is an eigenvalue of $M_f$ $\iff$ $\mu(f^{-1}(\lambda))>0$
\item[\df{71}] $\operatorname{ess-range} f = \left\{z\in\CC\,:\,\mu(f^{-1}(B_\eps(z))) > 0 \, \forall \eps>0\right\}$
\item[\ft{71}] $\sigma(M_f) = \operatorname{ess-range} f$
\item[\ft{71}] $\|(M_f - z)^{-1}\| = \Big(\operatorname{dist}\big(z,\sigma(M_f)\big)\Big)^{-1}$
\item[\ft{83}] if $T$ closed then $\rho(T)$ is open and $R_z = (T-z)^{-1}$ is analytic in $z$ on $\rho(T)$
\item[\df{85}] \emph{spectral radius}: $r(T) = \sup_{z \in \sigma(T)} |z|$
\item[\ft{85}] if $T$ bounded then $r(T) \le \|T\|$
\item[\ft{86}] $\sigma(A)\subset \RR$
\item[\ft{87}] $z\in\sigma(A)$ $\iff$ $\exists \{u_n\}\subset \cD(A) \text{ s.t.~} \|u_n\|=1 \text{ and } \|(A-z)u_n\| \to 0$
\item[\df{17}] $U$ is \emph{unitary} if it is bijective and an isometry (i.e.~$\|Ux\|=\|x\|\,\forall x\in\cH$)
\item[\ft{17}] $U$ unitary $\iff$ $U$ bijective \& $\ip{Ux}{Uy}=\ip{x}{y}\,\forall x,y\in\cH$
\item[\ft{102}] $U$ unitary $\iff$ $U\in\cL(\cH)$ and $UU^* = U^*U=I$
\item[\df{102}] \emph{functional calculus}: on $T$ we can apply a function $f:\CC\to\CC$ to create an operator $f(T)$
\item[\df{102}] $\SS = \{z\in\CC\,:\,|z|=1\}$ and $C(\SS)=\{f:\SS\to\CC\,|\,f \text{ is continuous (and periodic)}\}$
\item[\ft{103}] \textbf{continuous functional calculus for unitaries.} fix $U$ unitary. there is a map $C(\SS) \to \cL(\cH)$, $f\mapsto f(U)$ so that
\renewcommand{\labelenumi}{(\alph{enumi})}
\begin{enumerate}
\item[(0)] if $f(z) = 1$ then $f(U) = I$
\setcounter{enumi}{0}
\item $f(U)^*=\overline{f}(U)$
\item $f(U)g(U) = (fg)(U)$ \hfill {\footnotesize $\leftarrow$ \textbf{thus:} $f(U)g(U)=g(U)f(U)$}
\item if $f\ge 0$ then $f(U)\ge 0$
\item $\|f(U)\| = \sup_{z\in\SS} |f(z)|$
\end{enumerate}
\item[\df{105}] if $X$ is a metric space then $C(X) = \{f:X\to\CC \text{ continuous}\}$
\item[\df{105}] $\beta:C(X)\to\CC$ is \emph{positive} if $f\ge 0 \implies \beta(f)\ge 0$
\item[\ft{105}] \textbf{Riesz representation theorem.} suppose $X$ is a compact metric space and $\beta:C(X) \to \CC$ is linear and positive. there is a unique positive Borel measure on $X$ so that
	$$\beta(f) = \int_X f\,d\mu \qquad \forall f \in C(X)$$
\item[\df{105}] for $U$ unitary and $v\in\cH$ the \emph{spectral measure} is $\mu_v$ on $\SS$ so that $\ip{v}{f(U)v} = \int_{\SS} f\,d\mu_v$
\item[\ft{106}] for $\mu$ from the Riesz representation theorem, $C(X) \subset L^2(X,\mu)$ is dense
\item[\ft{107}] \textbf{spectral theorem for unitaries.} if $\cH$ is a separable Hilbert space and $U\in\cL(\cH)$ is unitary then there is a countable collection of finite measures $\nu_k$ on $\SS$, and a measurable space $(Y,\nu) = \cup_k (\SS,\nu_k)$, and a unitary map $W:L^2(Y,\nu) \to \cH$, so that
	$$W^{-1} U W = M_\eta$$
where $M_\eta \in \cL(L^2(Y,\nu))$ is a bounded multiplication operator and $\eta:Y\to \CC$ is equal to $\eta(z)=z$ on each copy of $\SS$
\item[\df{108}] the \emph{Cayley transform} $\ds \gamma(z) = \frac{z-i}{z+i}$ maps $\RR$ to $\SS$
\item[\ft{108}] \textbf{spectral theorem (multiplication operator form).} if $\cH$ is a separable Hilbert space and $A$ is self-adjoint on $\cH$ then there is a countable collection of finite Borel measures $\mu_k$ on $\RR$, and a measurable space $(X,\mu) = \cup_k (\RR,\mu_k)$, and a unitary map $Q:L^2(X,\mu) \to \cH$, so that
	$$Q^{-1} A Q = M_\alpha$$
where $M_\alpha \in \cL(L^2(X,\mu))$ is a (generally unbounded) multiplication operator and $\alpha:X\to \RR$ is equal to $\alpha(x)=x$ on each copy of $\RR$
\end{itemize}
\end{document}
