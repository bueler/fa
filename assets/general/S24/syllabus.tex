\documentclass[12pt]{article}

% Layout.
\usepackage[top=1.2in, bottom=0.9in, left=1.2in, right=1.2in, headheight=1in, headsep=6pt]{geometry}

% Fonts.
\usepackage{mathptmx}
\usepackage[scaled=1.0]{helvet}
\renewcommand{\emph}[1]{\textsf{\textbf{#1}}}

% Misc packages.
\usepackage{amsmath,amssymb,latexsym}
\usepackage{graphicx,hyperref}
\usepackage{array}
\usepackage{xcolor}
\usepackage{multicol}
\usepackage{tabularx,colortbl}
\usepackage{enumitem}

\hypersetup{
    colorlinks=true,
    linkcolor=blue,
    filecolor=magenta,      
    urlcolor=blue,
    pdfauthor={Ed Bueler}
    pdftitle={Syllabus for MATH F617 Spring 2024},
    }

% Paragraph spacing
\parindent 0pt
\parskip 6pt plus 1pt
\def\tableindent{\hskip 0.5 in}
\def\ts{\hskip 1.5 em}

\usepackage{fancyhdr}
\pagestyle{fancy} 
%\chead{\large\sf\textbf{}}
\lhead{\large\sf\textbf{Syllabus MATH F617}}
\rhead{\large\sf\textbf{Spring 2024}}
  
\newcommand{\localhead}[1]{\par\smallskip\textbf{#1} \smallskip\nobreak\\}%
\def\heading#1{\localhead{\large\emph{#1}}}
\def\subheading#1{\localhead{\emph{#1}}}

\newenvironment{clist}%
{\bgroup\parskip 0pt\begin{list}{$\bullet$}{\partopsep 4pt\topsep 0pt\itemsep -2pt}}%
{\end{list}\egroup}%


\begin{document}

\strut\par%\vskip-12pt
\heading{Essential Information}

%\vskip -12pt
\strut\hbox to \hsize{\tableindent\vtop{\halign{#\hfill\ts&#\hfil\cr
{\emph{Course Title}} & {\Large Functional Analysis} \cr
\strut & \cr
{\emph{Instructor}} & Ed Bueler \quad \href{mailto:elbueler@alaska.edu}{\texttt{elbueler\@@alaska.edu}} \cr
\strut & \cr
{\emph{Class meeting}} & MWF 2:15--3:15 pm, Chapman 107 \cr
\strut & \cr
{\emph{CRNs}} & in-person:\, 35367 \quad online: 35378 \href{https://canvas.alaska.edu/courses/18441}{(zoom link here)}\cr
\strut & \cr
{\emph{Public website}} & \href{https://bueler.github.io/fa/}{\texttt{bueler.github.io/fa}}\cr
\strut & \cr
{\emph{Canvas website}} & \href{https://canvas.alaska.edu/courses/18441}{\texttt{canvas.alaska.edu/courses/18441}} \cr
\strut & \cr
\emph{Required text} & D.~Borthwick, \textsl{Spectral Theory}, Graduate Texts in\cr
 & Mathematics 284, Springer 2020\cr
}
\hfil}}


\heading{Description}
Functional analysis is the theory of infinite-dimensional vector spaces and linear maps upon them. It is the mathematical home of partial differential equations, boundary value problems, quantum mechanics, the finite element method, field theories (electromagnetic and gravitational), fluid mechanics, and signal processing.

The UAF catalog description of the course is this:  \textsl{Study of Banach and Hilbert spaces, and continuous linear maps between them.  Linear functionals and the Hahn-Banach theorem. Applications of the Baire Category theorem.  Compact operators, self adjoint operators, and their spectral properties. Weak topology and its applications.}

The above description is reasonably accurate, and all of these topics will arise.  However, I will strongly emphasize some parts and gloss others.  The course will focus on:  \textsl{Hilbert spaces including Sobolev spaces.  Continuous and unbounded operators (linear maps) on Hilbert spaces.  The spectral theory of compact and self-adjoint operators, including the Laplacian.  Connections with finite-dimensional linear algebra and partial differential equations.  Motiviation of the axioms and basic concepts of quantum mechanics.}

Mathematically-inclined students from the sciences and engineering are encouraged to register, as are graduate students in mathematics looking for an elective with practical relevance.  This course is particularly aimed at students interested in the mathematics of quantum mechanics and partial differential equations.


\clearpage\newpage
\phantom{foo}
\heading{Prerequisites}
Officially: \textsl{MATH F314 Linear Algebra and MATH 401 Introduction to Real Analysis, or permission of instructor. Recommended: MATH F422 Introduction to Complex Analysis and MATH F641 Real Analysis.}

For graduate students with a background from another university, these prerequisites describe rigorous, though introductory, exposure to the analysis of real functions, plus exposure to linear algebra with a little rigor.  The basics of complex numbers are assumed.  Quite a bit of mathematical maturity and motivation is assumed.  The difficulty of assigned work will depend on the student's comfort doing proofs, especially with real functions, convergence, continuity, and integrals.


\heading{The Hybrid Classroom}
There are two sections of the class, in-person (901; crn 35367) and online (701; crn 35378).  They are treated as one course and occur simultaneously.  In this ``hybrid'' set-up, each lecture will be a recorded Zoom session generated from Chapman 107.  (The link for the Zoom session is \href{https://canvas.alaska.edu/courses/18441}{in Canvas}.  The recordings will be linked from inside Canvas only; they are not public.)  I will try to treat all students the same regarding proctored assessments---see below---and participation during class time.  Students have certain obligations to help make this work:
\begin{itemize}
\item \textbf{in-person students}: Please participate as energetically as you can.  I prefer for in-person students to turn in their homework assignments on paper.
\item \textbf{online students}:  Please sign into the Zoom session, from Canvas, just before class starts.  Please participate as energetically as you can, and, if possible, keep your camera on.  Regarding in-class group work, check for worksheet PDFs from the \href{https://bueler.github.io/fa/}{public site} before class starts.  When you turn in homework assignments electronically, please generate clear, well-ordered, and combined PDFs; this may require scanning documents.  You will need to schedule proctoring for the in-class assessments (see below), or attend in person on those days.
\end{itemize}


\heading{Schedule and Online Materials}
The \href{https://bueler.github.io/fa/}{public course website} includes a \href{https://bueler.github.io/nla/assets/general/S24/schedule.pdf}{day-by-day schedule} listing the textbook sections to be covered during each lecture, the due date of each homework Assignment, and the dates for the Midterm Quizzes and Final Exam.  Please consult this schedule frequently; it is subject to change and will be kept up to date.

Most course materials (syllabus, schedule, homework Assignments, code examples, etc.) will be posted on the \href{https://bueler.github.io/fa/}{public website}.  Some course materials (student grades, homework and exam solutions) will go on the \href{https://canvas.alaska.edu/courses/18441}{Canvas site}.


\clearpage\newpage
\phantom{foo}
\heading{Office Hours and Communication}
My office hours are shown online at \href{http://bueler.github.io/OffHrs.htm}{\texttt{bueler.github.io/OffHrs.htm}}; I hold them in Chapman 306C.  Students can also schedule meetings with me outside of these hours.  I will use Canvas to send announcements.  If I need to contact you outside of class times, I'll try to email via Canvas.  (Please set your email address in Canvas to one that you check regularly!)


\heading{Evaluation and Grades}
\vskip -10pt

\begin{tabular}{|l|l|r|}
\hline
Homework & nearly weekly & 50\% \\
\hline
Midterm Quiz 1 & in-class Wednesday February 28 & 15\%  \\
\hline
Midterm Quiz 2 & in-class Wednesday April 3 & 15\%  \\
\hline
Final Exam     & in-class Friday May 3 & 20\% \\
\hline
total & & 100\% \\
\hline
\end{tabular}

Scores for specific assessments may be adjusted based on the actual difficulty of the work and/or on average class performance, and adjustments will be applied to all students equally.  The scores of the various parts will be summed and the final course grade will be assigned as follows.

\begin{tabular}{llllll}
A  & 93--100\% & B- & 79--81\%  & D+ & 65--67\%  \\
A- & 90--92\%  & C+ & 76--78\%  & D  & 60--64\%  \\
B+ & 87--89\%  & C  & 68--75\%  & D- & 57--59\%  \\
B  & 82--86\%  & C- & not given & F  & $\le$ 56\%
\end{tabular}

These ranges are a guarantee and a lower bound.  I reserve the right to increase your grade above these ranges based on the actual difficulty of the work and/or on average class performance.  Any such increases will preserve grade ordering by weighted total score.


\heading{Homework}
Homework is due at the start of class.  \emph{Late homework is not accepted.}  If you have unavoidable circumstances which do not allow you to turn in an Assignment on time then please contact me (\href{mailto:elbueler@alaska.edu}{\texttt{elbueler\@@alaska.edu}}) in advance.

The homework consists of proofs, rigorously-justified examples and counter-examples, and by-hand computations.  Occasionally a computer visualization will arise, but no particular facility with programming is needed.  Problems very similar to, or shortened versions of, Homework problems will appear on the in-class Midterm Quizzes.


\heading{Exams}
There will be two in-class, hour-long Midterm Quizzes.  These Midterms will firm-up the basic concepts and definitions addressed more thoroughly in the Homework.  The 2-hour, in-class Final Exam will require you to be familiar with, and prepared on, major themes and theorems in the course.  I will describe the format of the Final in more detail in due course.

Make-up Quizzes or Exam will be given only for documented extenuating circumstances, at my discretion.  Department policy (below) does not allow me to move the time of the Final Exam.


%\clearpage\newpage
\phantom{foo}
\heading{Rules and Policies}
\vskip -20pt

\subheading{Incomplete Grade} 
Incomplete (I) will only be given in
  DMS courses in cases where
  the student has completed the majority (normally all but the last
  three weeks) of a course with a grade of C or better, but for
  personal reasons beyond his/her control has been unable to complete
  the course during the regular term. Negligence or indifference are
  not acceptable reasons for granting an incomplete grade.

\subheading{Late Withdrawals} 
A withdrawal after the deadline from a DMS course will
  normally be granted only in cases where the student is performing
  satisfactorily (i.e., C or better) in a course, but has exceptional
  reasons, beyond his/her control, for being unable to complete the
  course.  These exceptional reasons should be detailed in writing to
  the instructor, Department Chair and the Dean.

\subheading{No Early Final Examinations}
Final examinations for DMS courses shall not be held earlier than the date and time published in the official term schedule.  Normally, a student will not be allowed to take a final exam early.  Exceptions can be made by individual instructors, but should only be allowed in exceptional circumstances and in a manner which doesn't endanger the security of the exam.

\subheading{Academic Dishonesty}
Academic dishonesty, including cheating and plagiarism, will not be tolerated.  It is a violation of the Student Code of Conduct and will be punished according to UAF procedures.

\subheading{Student protections and service statement}
Every qualified student is welcome in my classroom.  As needed, I am happy to work with you, Disability Services, Veterans' Services, Rural Student Services, and so on, to find reasonable accommodations.  Students at this University are protected against sexual harassment and discrimination (Title IX), and minors have additional protections.  For more information on your rights as a student and the resources available to you to resolve problems, please go the following site: \href{https://www.uaf.edu/handbook/}{\texttt{www.uaf.edu/handbook}}.

\hfill  \scriptsize [syllabus version: \today] \normalsize

\end{document}
